% -*- fill-column: 72 -*-

\section{The Basic Commutativity Rule}
\label{sec:rule}

We now precisely present the basic commutativity rule and sketch a
proof of its main claim, namely that commutative operations can be made
to scale.
%
The rule depends intimately on \emph{state} and \emph{history}.
%
It says that whenever operations commute \emph{in the context of some
trace} of operations, there exist implementations of those operations
the scale \emph{in the same context}.
%
We argue that this history dependence captures an important property of
practical scalability, and discuss how the definition could be
broadened.

\begin{comment}
To give a sense for how the connection between commutativity and
scalability might be formalized, this section provides a preliminary
definition of scalability, commutativity, and a rule that connects the
two.  While this makes our proposal more concrete, we hope to further
improve on these definitions and rules to make them easier to apply
in practice.

In this section, we start by first defining an abstract machine designed
to capture the primary factors in scalability on real hardware.
Then, we define commutativity in a way
that applies well to system calls with state-dependent behavior.  Finally,
our hypothesis connects the commutativity of interfaces with the
existence of implementations that scale linearly on the abstract
machine.
\end{comment}

\subsection{Definitions}
\label{pf}

\def\OP#1{\textsl{#1}}
\def\INV#1#2#3{\text{\OP{#1}}_{#3}(#2)}
\def\RES#1#2#3{\text{\OP{#1}}_{#3} \rightarrow #2}
\def\HRESTRICT#1#2{#1|#2}
\def\HIST#1{#1}
\def\LEGALHIST{\mathcal{H}}
\def\PARTIALHIST{\mathcal{P}}
\def\STEP{\rightarrow}
\def\TRUE{\textsc{true}}
\def\FALSE{\textsc{false}}
\def\OK{\textsc{ok}}
\def\FAIL{\textsc{fail}}
\def\qed{$\square$}

\begin{comment}
The abstract machine gives us a way to reason about when a particular
implementation scales, which, in turn, lets us weigh potential
implementations of an given interface.
In order to reason about when an interface's implementation \emph{can
  be} scalable, this section turns to \emph{commutativity}, an interface
design property that captures the inherent communication needs of an
interface's implementations.
\end{comment}

Informally, operations commute when 
they produce the same results regardless of their execution order
and later operations cannot reveal their execution order.  To
formalize this, we build on the terminology of Herlihy and
Wing~\cite{herlihy:linearizability}.


An \emph{operation} is the pair of two \emph{actions}, an
\emph{invocation} and a \emph{response}. In the context of an
operating system kernel, the invocation is basically a system call,
including arguments, and the response is the return value. Each
operation takes place on a single \emph{thread}. We write the thread
as a subscript. An example invocation is written
$\INV{operation}{\text{arguments}}{\text{thread}}$, and its response
as $\RES{operation}{\text{response}}{\text{thread}}$.

A \emph{serial history} is a sequence of actions in which each
invocation is immediately followed by its response.

A \emph{history} $\HIST{H}$ is a sequence of actions where for any
thread $t$, the subsequence formed by selecting just operations on $t$
is serial. (We write this subsequence as $\HRESTRICT{\HIST{H}}{t}$.)
%
This means that each thread's actions look like the trace of
some single-threaded program, alternating between invocation and
response. However, operations on \emph{different} threads can overlap.
%
A history may contain \emph{outstanding} invocations---i.e., invocations
that lack corresponding responses. However, since thread subhistories
are serial, there can be at most one outstanding invocation per thread,
and that outstanding invocation must be the last invocation for that
thread.

\emph{Legal} histories capture the notion of correctness. Some
histories are legal and some aren't. A legal history is allowed by the
specification of the objects under study (e.g., the specifications
of systems calls); an illegal history isn't
allowed. If a history is legal, then every prefix of that history is
also legal. The (prefix-closed) set of legal histories is called
$\LEGALHIST$.

We assume that $\LEGALHIST$ exists, but aren't
concerned with how it is created. For example, imagine a
simple interface for a single shared variable with two
functions: $\OP{put}(x)$, which sets the shared variable's value to $x$,
and $\OP{get}()$, which returns the
variable's current value as its response.  $H_1$ and $H_2$ below are
legal histories, but $H_3$ is not (because $\OP{get}_1$ returns an
unexpected value). $H_2$ shows an example of overlapping
operations.
%
\begin{align*}
  H_1 = [&\INV{put}{1}{1}, \RES{put}{\OK}{1}, \INV{get}{}{1},
  \RES{get}{1}{1}] \\
  H_2 = [&\INV{put}{1}{1}, \INV{put}{2}{2}, \INV{get}{}{3},
  \RES{put}{\OK}{2}, \RES{get}{1}{3}, \RES{put}{\OK}{1}] \\
  H_3 = [&\INV{put}{1}{1}, \RES{put}{\OK}{1}, \INV{get}{}{1},
  \RES{get}{0}{1}]
\end{align*}

An \emph{implementation}'s job is to generate responses for outstanding
invocations.
%
We model implementations using a \emph{step relation} over histories,
written $\STEP$.
%
The step relation completes outstanding invocations: if $P \STEP P\cdot
r$, then $P$ is a history with one or more outstanding invocations, and
$r$ is a response to one of the outstanding invocations.
%
Different implementations might generate different responses, and
responses need not be unique (a single implementation might allow both
$P \STEP P\cdot r$ and $P \STEP P\cdot r'$ with $r \neq r'$).
%
Implementations are modeled as state machines that work on a shared
memory.
%
Each thread has a separate state machine that receives invocations on
that thread as externally generated events.
%
Although a history imposes a specific order on all invocations and
responses, threads can in general observe that order only by
writing and reading shared memory.
%
We leave full discussion of this model for future work, but
discuss its preliminary development in Section~\ref{sec:abscal}.

A set of legal histories defines both an interface and correctness
requirements for implementations of that interface.
%
A \emph{correct} implementation will ensure that every generated response is
legal.
%
Specifically, an implementation is correct if, whenever $P \in
\LEGALHIST$ and  $P \STEP P\cdot r$, we also have $P\cdot r \in
\LEGALHIST$.

Note that an implementation has no say over the order in which
\emph{invocations} arrive.
%
This depends only on the set of legal histories.
%
In most situations, such as operating systems kernels, it's important
that the set of legal histories allow any invocation at any time (though
of course an implementation would generate error responses for many
invocations, such as requests to read from non-existent file
descriptors).
%
We elide this requirement for this presentation, as well as some requirements concerning blocking
interfaces---additional work is needed to ensure that implementations
don't block forever unless that is appropriate.


Implementation correctness can imply the existence of state
that implementations must keep.
%
For instance, since $\OP{get}()$ above is required to return
the most recently $\OP{put}$ value, any correct implementation must keep
some state corresponding to that value (although this state need not be
a single shared memory location).


\begin{comment}
\textbf{Nondeterminism.} We assume without loss of
generality${}^{\text{????}}$ that all operations have deterministic
return values. Of course, real operations often return nondeterministic
results. This can be modeled in various ways. For example, we might
model a nondeterministic action as returning a \emph{set} of values,
where the set contains all legal return values.
\XXX[AC][Do we need this, given that legality already accounts for
nondeterminism?]
\XXX[E][Yes, because commutativity doesn't say just that the scrambled
history is legal with SOME return values, we say that the scrambled
history is legal with THE SAME RETURN VALUES as in the original trace]
\XXX[AC][That's exactly my point.  Our definition \emph{already}
captures this, so why do we need to say something fuzzy about making
interface deterministic?  If we're going to say something about
nondeterminism, shouldn't we draw the reader's attention to how our
definitions capture it, rather than saying that they don't without
additional modeling?]
\XXX[E][I don't think it is your point but I don't think we actually need
to care about nondeterminism at this point in the doc]
\end{comment}

\subsection{The rule}
\label{rulerule}

We are now ready to define commutativity and the basic commutativity
rule.

Given a series of actions $X$,
a series of actions $Y$ is an
\emph{interleaving} of $X$ if, for every thread $t$,
$\HRESTRICT{Y}{t} = \HRESTRICT{X}{t}$.
%
This means that $Y$ and $X$ contain exactly the same actions, and that
their per-thread subsequences are identical. However, it allows $Y$ to
reorder actions \emph{among threads} relative to $X$.
%
A series of actions $Z$ is a \emph{sub-interleaving} of $X$ if $Z$ is a
\emph{prefix} of some interleaving of $X$.
%
This means that, for every thread $t$, $\HRESTRICT{Z}{t}$ is a prefix of
$\HRESTRICT{X}{t}$. Sub-interleavings, like interleavings, preserve per-thread
orders of actions in $X$; unlike interleavings, they are allowed to leave
later actions out.
%
Any $X$ is both an interleaving and a sub-interleaving of itself.

Consider a legal history $\HIST{H}$ and a contiguous subsequence $X$ of its
actions, with $\HIST{H} = A \cdot X \cdot B$.
%
We say that $X$ \emph{commutes} in $\HIST{H}$ if, given any
sub-interleaving $Z$ of $X$, any $B'$ so that
$A \cdot Z \cdot B'$ is legal, and any interleaving $Z'$ of $Z$, the
history $A \cdot Z' \cdot B'$ is also legal.
%
Thus, when $X$ commutes in $\HIST{H}$, any actions after $X$ can't
distinguish between different orders of the actions in $X$.

The \textbf{basic commutativity rule} is then stated as follows.
%
Assume there exists a correct implementation over some set of legal
histories $\LEGALHIST$.
%
Take some $\HIST{H} \in \LEGALHIST$ and some $X$ that commutes in
$\HIST{H}$.
%
Then there exists a correct implementation over $\LEGALHIST$
that is \emph{scalable} over the operations wholly
contained in $X$.
%
Here, ``scalable'' means that operations on different threads do not
communicate using shared memory: a memory location written by the
implementation of thread $t$ in $X$ is not read by any other thread in
$X$.


\subsection{Proof sketch}

Why is the basic commutativity rule true?
%
The intuition is simple.
%
Since $X$ commutes in $\HIST{H}$, the implementations of the
invocations in $X$ must not need to communicate: they return the same
results regardless of order, and later functions in $H$ can't
distinguish the order either.

This intuition can be formalized by \emph{constructing} an
implementation $M$
whose threads don't communicate in $X$.
%
This implementation is tailored for a single history and
commutative region. Though it is \emph{correct} for every history, it is
guaranteed to \emph{scale} only for $H$ and $X$.
%
$M$ compares received invocations against a copy of $H$
embedded in $M$.
%
If the received invocations match $H$,
then (at least so far) the operation sequence is the one
$M$ is tailored for, and $M$
plays back responses from its copy of $H$.
%
If they don't match, then $M$ is seeing a
sequence of operations other than the one it's tailored for,
and $M$ switches over to
simulating a correct reference implementation, $R$.
%
This requires replaying the previous invocations to prepare $R$'s
internal state.

Since it is simply playing back responses, you might
expect $M$ to scale over all of $H$.
%
It doesn't, though, since tracking received invocations is not scalable.
%
In general global synchronization is required to ensure that all
threads' received invocations match those in $H$---\emph{except} in a
commutative region.
%
There, by definition, \emph{any} replay order must create an equivalent
result. Thus, $M$ can switch to unsynchronized, scalable tracking upon
reaching $X$, which, as it knows, commutes.

Without further assumptions on the nature of the interface being
represented, a specifically tailored implementation is the best we can
do.
%
We will investigate assumptions that allow us to construct more generic
scalable implementations as part of our proposed work.
%
A more formal presentation of this argument follows.

By assumption a correct reference implementation $R$ exists.
%
Consider a legal history  $\HIST{H} = A \cdot X$ where $X$ commutes in
$\HIST{H}$.
%
Assume without loss of generality that, when given the invocations in
$H$, the reference implementation always generates the responses
indicated by $H$.
%
% Then an implementation that scales over $X$ when given a history
% beginning with $A\cdot X$, but returns the same result as the reference
% implementation for \emph{any} history, can be constructed by tracking
% received invocations against $A\cdot X$.
% %
% Within $A$, invocation tracking must be non-scalable; but within $X$, it
% can be scalable, thanks to commutativity.
% %
% When received invocations either complete $A\cdot X$ or diverge from it,
% the constructed implementation replays the received invocations to the
% reference implementation.
%
Then construct $M$ by adding to $R$'s state one history and one history
program counter per thread.
%
For thread $t$, we store $H_t = A \cdot
  (\HRESTRICT{X}{t})$ and $\textit{pc}_t$. Each $\textit{pc}_t$
  initially points at the first invocation in $H_t$. The special value
  $\textit{pc}_t = \FAIL$ indicates the thread should fall back to the
  reference implementation.
%
The implementation of thread $t$ is as follows.
  \begin{CompactEnumerate}
\item If $\textit{pc}_t = \FAIL$, or $\textit{pc}_t$ points beyond
  $H_t$, then $t$ falls back to the reference implementation.
\item If $\textit{pc}_t$ points at an invocation or response not in
  thread $t$, then $t$ waits for another thread to advance
  $\textit{pc}_t$ or set it to \FAIL.
\item If invocation $i$ is received, and $\textit{pc}_t$ points to an
  invocation $I$ on thread $t$, then:
  \begin{CompactEnumerate}
\item If $i \neq I$, then the thread non-scalably falls back to the
  reference implementation. This involves pausing all other threads, taking
  a snapshot of all $\textit{pc}_t$ values, replaying the
  invocations indicated by the \textit{pc}s on the backup
  implementation's state, and then setting all $\textit{pc}_t$ values to
  \FAIL.
\item If $i = I$ and $\textit{pc}_t$ is in $A$, then the thread
  non-scalably increments \emph{all} $\textit{pc}$ values.
\item If $i = I$ and $\textit{pc}_t$ is in $\HRESTRICT{X}{t}$, then the
  thread scalably increments \emph{only its own} $\textit{pc}_t$.
  \end{CompactEnumerate}
\item If $\textit{pc}_t$ points to a response on thread $t$, then the
  implementation generates that response and increments $\textit{pc}_t$.
  If this response is in $A$, the implementation also
  non-scalably increments the other threads' \textit{pc} values.
\end{CompactEnumerate}

\noindent
%
This implementation
%
exactly replays the responses in $A$ using lockstep \textit{pc}
updates; these responses are legal by assumption (legal histories are
prefix-closed).
%
Over $X$, the order of responses generated by this implementation may
differ from that of $H$, but any such order is legal because of
commutativity.
%
If and when the input history diverges from $H$, the implementation
replays the received invocations to the reference implementation.
%
For invocations in the commutative region $X$, the replay order may
differ from the actual order in which the original invocations were received, because a scalable
implementation cannot record a global reception order.
%
Again commutativity shows that any replay order is allowed, since later
invocations can't distinguish among them.
%
Thus, thanks partially to commutativity, $M$ both scales over $X$ and is
correct for any sequence of invocations. \qed

\subsection{Examples}

The implementation constructed in the proof sketch is clearly not one
anyone would use.
%
Nevertheless, the definition of commutativity on which it relies
captures the implicit interface structure that allows for better
scalable implementations.
%
Some concrete examples of commutative interfaces will show both more
realistic scalable implementations of commutative interfaces, and
conceptual benefits of our history-dependent definition of
commutativity.


\paragraph{Reference counter}

Consider a reference counter with three operations, \OP{inc}, \OP{dec},
and \OP{isdead}.
%
The \OP{inc} operation increments the reference count, returning
nothing.
%
The \OP{dec} operation decrements the reference count, returning
nothing.
%
Finally, the \OP{isdead} operation returns \TRUE\ iff the count is
non-positive.
%
The intent is that a client calls \OP{inc} when obtaining a reference
and \OP{dec} when releasing a reference. After \OP{dec}, the client
calls \OP{isdead}, freeing the object if \OP{isdead} returns \TRUE.

Consider the following history, assuming that the initial value of the
count is 3:
%
\begin{align*}
\HIST{H} = [& \INV{isdead}{}{1}, \RES{isdead}{\FALSE}{1},
	\INV{isdead}{}{2}, \RES{isdead}{\FALSE}{2}, \\
	&\INV{dec}{}{3}, \RES{dec}{\OK}{3},
	\INV{dec}{}{4}, \RES{dec}{\OK}{4},
	\INV{dec}{}{5}, \RES{dec}{\OK}{5}, \\
        &\INV{isdead}{}{6}, \RES{isdead}{\TRUE}{6}].
\end{align*}
%
Each operation is
performed by a different thread, indicated by subscript.
%
The following subsequences commute in $\HIST{H}$.

\begin{CompactItemize}
\item Each individual invocation--response pair obviously commutes, since an
  individual invocation--response pair has no nontrivial re-interleavings.
  This doesn't tell us anything, since the scalability of an
  individual operation in isolation is meaningless.
\item The first two \OP{isdead} calls commute. They can be made to scale
  by keeping per-core counter snapshots, and forcing \OP{dec} to update
  each core's snapshots.
\item The three \OP{dec} calls commute. They can be made to scale by
  keeping per-core counters, and forcing \OP{isdead} to sum all per-core
  counters before returning a result.
\item The first four operations commute (two \OP{isdead} and two \OP{dec}). These can be
  made to scale with a combination of per-core counters (whose sum is one
  less than the true counter) and
  per-core ``\OP{isdead}'' snapshots. An ``\OP{isdead}'' snapshot is defined to
  \TRUE\ iff the sum of the per-core counters is negative. A \OP{dec}
  call decrements its per-core counter. If the per-core counter goes
  negative, the \OP{dec} call must read and sum all the counters and update
  the ``\OP{isdead}'' snapshots accordingly.
  In this history, threads 3 and 4 might start out with per-core
  counters equal to 1, with thread 5's per-core counter equal to 0.
  Then the first four operations will execute scalably.\footnote{%
  %
  The operations would not execute scalably given other
  initial per-core counter assignments.}
\end{CompactItemize}

The first \emph{five} operations do \emph{not} commute in $\HIST{H}$.

Now, the three different commutative subsequences above call for three
\emph{fundamentally different} scalable implementations. This is exactly
right! We know that a read-mostly counter can be made scalable, and a
write-mostly counter can be made scalable, though those implementations
are very different. In practice, an implementer would decide which
\emph{kinds} of commutative subsequences are more common, and write an
implementation scalable in those situations.


\paragraph{Reference counter interfaces}

Typical reference counter interfaces differ from the one above by
combining \OP{isdead} and \OP{dec} into one call, \OP{deccheck}, where
\OP{deccheck} both decrements the reference count and returns
\TRUE\ if the decrement caused the counter to become non-positive.
%
(For instance, Linux's \verb+atomic_inc+ and \verb+atomic_dec_and_test+
functions provide this kind of interface.)
%
Commutativity reasoning shows how this interface choice limits
scalability.
%
The analogous history would be as follows:
%
\begin{align*}
\HIST{H'} = [&\INV{deccheck}{}{3}, \RES{deccheck}{\FALSE}{3},
	\INV{deccheck}{}{4}, \RES{deccheck}{\FALSE}{4}, \\
&\INV{deccheck}{}{5}, \RES{deccheck}{\TRUE}{5}]
\end{align*}
%
Now, unlike in $H$ above, the three \OP{deccheck} calls \emph{do not}
commute.
%
The interface with combined decrement is harder to make scale than
the interface that separates decrement from checking alone.
%
This can lead to new implementation techniques.
%
For example, in the separated interface, perhaps
calls to \OP{isdead} could be moved to a background
thread that occasionally collected garbage, allowing \OP{inc} and
\OP{dec} calls to scale.
%
On the other hand, if, in most histories, the counter has a large
positive value, then even the \OP{deccheck} interface commutes, and
therefore can be made to scale using per-core counters.

Basic commutativity reasoning points out where scalability is possible.
%
Other forms of reasoning are required to choose which kinds of
scalability are important to implement.
%
We envision developing a \emph{set} of commutativity rules that
distinguish interfaces by \emph{how easily} they commute---in other
words, that distinguish interfaces that commute only with specific
arguments or at specific points in a history from interfaces that
commute more generally.


\begin{comment}
Careful analysis using our commutativity definition reveals that
increments and decrements commute as long as the counter is large
enough. If the counter is small, however, decrements may not commute.
For example, if the counter is one, and there are two concurrent
decrements, which one returns true and which false depends on
the order. Thus this interface does not allow a generally commutative
implementation.

However, the commutativity analysis suggests a way to change the
interface so that it can scale. In particular:
\begin{verbatim}
  inc() -> void
  dec() -> void
  nonzero?() -> boolean
\end{verbatim}
Now increments and decrements scale. If the calls to {\tt nonzero?}
can be moved to a background thread that occasionally collects
garbage, then the common case can be made to scale.

\XXX[rtm][this is a lot like the ispos example, but cast in a way
that I think makes it more clear why it's useful to the programmer to
have a precise definition of commutativity. also I think we should have
an example that involves a programming using commutativity to help
choose a better interface.]
\end{comment}


\paragraph{File creation}  Consider the following history, in which
threads in different processes create different files in the same directory:
%
\begin{align*}
\HIST{H} = [& \ldots, \INV{open}{\text{\code{"/tmp/a", O_CREAT|O_EXCL}}}{1},
    \RES{open}{\textit{fd1}}{1}, \\
  & \INV{open}{\text{\code{"/tmp/b", O_CREAT|O_EXCL}}}{2},
    \RES{open}{\textit{fd2}}{2}, \ldots ].
\end{align*}
%
This is the file-creation example mentioned earlier.
%
The subsequence containing the four responses and invocations shown
above is commutative in $\HIST{H}$.
%
These operations both write
the directory, so it's natural to assume that they must modify shared
state. An implementer might naturally throw up her hands and not even
try to make them scale. But ignoring quotas (and assuming inode
numbers are nondeterministic, and that directory listings return values
in random order, etc.), the operations commute. Later code
can't distinguish the order in which the operations completed. This is
a hint that a scalable implementation might exist. As in fact one
does---given a hash-based directory definition, cores creating
``/tmp/a'' and ``/tmp/b'' could do so without
communicating. Even updating the directory modification time could be
made to scale by storing per-core modification times to be resolved on
read, or when the disk is actually written.
%
On the other hand, if the threads were in \emph{the same} process, the
operations wouldn't commute, because different interleavings of the
operations would result in different returned file descriptors.


\begin{comment}
% Eddie doesn't get the point of this and so doesn't want to fix it to
% be correct

\paragraph{Delayed allocation} You're designing an allocator.
It may run out of resources. You consider an interface design in which
the allocator returns an error indication if there are no free
resources. However, concurrent calls to the allocator therefor won't
commute, suggesting that there may not be a scalable implementation. You
think ``why not just have the allocator delay until resources become
available?'' With intuitive notions of commutativity, this appears to
make the allocator commutative, since there's no obvious difference
between a call when there were free resources and a call when there
weren't.
%
However, once you apply our careful definition of commutativity, you'll
realize that the delaying allocator doesn't make allocation commute.
Both the invocation and the response have to be in the re-interleaving
window. The definition says that every re-interleaving has to leave a legal
$R$. But you can't re-interleave the return of the allocation call that had
to wait so that it goes before the return of the allocation call that
didn't.
\end{comment}

\subsection{Discussion}

The basic commutativity rule helps separate the job of a scalable
systems programmer into
two problems:

\begin{CompactItemize}

\item When designing an interface, the designer can decide if a given
  interface design could scale without first having to implement it,
  by reasoning about commutativity at the level of the interface's
  specification.

\item Once the designer has decided on an interface and which operations
  commute, the designer can think about an implementation that is
  more efficient than the general construction given above by exploiting the
  semantics of the specific interface.

\end{CompactItemize}

\noindent
In practice, designers make trade-offs between scalability
and other desired properties.  For example, memory is not
unlimited, so the kernel must reuse memory; but re-using memory that has been used
by an operation is not commutative with that operation. As another example, an
implementation that scales linearly may not be the
highest performing implementation for a particular core count.
Some implementations of a scalable reference counter may be inefficient for a
small number of cores.

Determining which operations commute can be non-trivial, as illustrated
by the file creation and counter examples above.  In the best case,
some operations are always commutative, such as multiple invocations
of \code{getpid()}.  For these cases, our hypothesis suggests that
the operations should always be scalable.  In other cases, as we've
shown, commutativity depends on system state (e.g., whether quotas are
enabled) or arguments (e.g., whether two \OP{open} calls are on different
threads of the same process); or there may be no maximal set of mutually
commutative actions (e.g., the reference counter), forcing the designer
to choose which operations to scale.
%%  , but is relatively
%% straightforward, as in the file creation example:
%% as long as quotas are not enabled, the
%% operations should scale linearly.  The most complex situations arise when
%% different subsets of operations commute but there is no single maximal set
%% of operations that commutes together.
%% A similar situation might arise if quotas are enabled in a file system,
%% or a pipe becomes full or empty. 
In these cases, an interface designer
may need to provide explicit hints as to which of the commutative sets
of operations they decided to make scalable, or they may opt to change
the interface.
