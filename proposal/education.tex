\section{Curriculum Development Activities}
\label{sec:education}

The educational goal of this proposal is to enable advanced systems classes to
teach undergraduate and graduate students about multicore systems programming.
Based on our personal experience we believe that this topic can be best taught
and learned through programming labs, since the issues are subtle and there are
no general strategies to parallelize systems.
%
The recent textbook \emph{The Art of Multiprocessor
Programming}~\cite{herlihy:art} does a great job of documenting a library of
algorithms, but doesn't teach students how to design their software for
parallelism or teach which data structure to use when.  We will create such labs
using \sys and make \sys and the labs publicly available. Our initial targets are
advanced systems classes at Harvard and MIT (CS 261 and CS 264 at Harvard,
6.828 at MIT).

The development of \sys\ for advanced classes will follow our approach with
xv6~\cite{xv6}. As part of previous NSF
grants, the MIT PIs developed a new multiprocessor teaching operating system, xv6, along
with new labs.  We also have written a commentary, inspired by Lions, for xv6.
Now xv6 is used in several schools around the world for teaching, as well as
advanced undergraduate and graduate projects.  As a traditional
multiprocessor operating system with big locks, xv6 is unsuitable for teaching
about parallel programming for many cores.  To fill that gap, we will make \sys
suitable for classes and research projects, following the same road as we have
done for xv6.

We will initially experiment with \sys\ in classes by using it in MIT's operating
system class, 6.828.  The second half of 6.828 is project
based, and we will encourage students to do projects in the context of \sys.
Since the high-level structure of xv6 and \sys are similar, students should be
able to get of the ground quickly.  In addition, we will continue to incorporate
more advanced parallel programming material into 6.828 to give students the
necessary background to execute projects using \sys.  It is not uncommon for
these projects to grow into advanced undergraduate projects to obtain the
bachelor degree or a master thesis to obtain the Meng degree.

The advanced OS class taught at Harvard uses a different system, JOS,
which has been developed by both groups of PIs for many
years.
%
Unlike xv6, JOS is incomplete on its own: students make it work by
adding code.
%
JOS's simplicity makes it a good laboratory for experimenting with
commutative interfaces.

We plan to teach at least one research seminar class on parallel
systems programming using \sys\ and/or a parallel JOS\@.
%
This class will be taught at Harvard, cross-registered at MIT\@.

The projects that we are planning involve studying different subsystems and
applications, and designing the best scalable implementation for that subsystem
or application. For example, a project may involve designing a VM system with
different concurrent data structures~\cite{herlihy:art} and evaluating which one
is the most suitable for a VM system.  Another project may involve designing and
implementing a scalable parallel garbage collector. The latter is also a good
driver for stressing the VM system, since the garbage collector must interact
frequently with the operating system to acquire and release memory.

Of course, this proposal itself is an opportunity for teaching graduate and
undergraduate students about how to do research in multicore systems.  The
proposal will support mostly graduate students with whom he will work one on
one.  Through bachelor and master projects we expect that a number of
undergraduates to be involved as well. For instance, two MIT undergraduates are
involved in the development of \sys for this proposal.

Through this combined approach of teaching and research we expect to train a
cadre of students at Harvard and MIT about multicore systems programming. By
making all material publicly available we hope that other universities can use
the \sys infrastructure to train students.  Given the need in industry for
students who understand advanced parallel systems programming, we are hopeful
that \sys could see significant adoption at other universities.

% LocalWords:  multicore PIs MSRA Kaashoek EECS mentorship WTP OCW JOS
% LocalWords:  adoptable
