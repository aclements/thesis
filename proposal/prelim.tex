\section{Preliminary results}
\label{s:prelim}

\subsection{Abstract machine}
\label{sec:abscal}

The commutativity rule and proof sketch make assumptions about the
circumstances under which hardware operations scale. In summary, they
assume an ``abstract machine model'' as follows. Each core effectively
has an infinite local cache that is kept consistent with other local
caches using a coherence protocol. A set of operations scales as long
as the set of memory locations written to by each core is disjoint
from the set of memory locations read from or written to by all other
cores. That is, for every pair of cores, the two cores access either
completely disjoint memory locations, or if they do share any memory,
both cores only read the shared memory. As a result, a set of
operations that scales will not experience coherence misses; rather,
it will only experience cache misses it would have experienced when
executing in isolation or on a single core.

This abstract machine is simpler than any actual machine; it makes few
assumptions about real hardware performance models, akin to how
asymptotic analysis models performance in a largely
hardware-independent way. Though the abstract machine does not always
agree with real hardware, we have experimentally verified agreement in many
regimes. \XXX[Austin][Include results.]
Roughly speaking, a real machine fits the abstract model as long as a
workload's working set fits in cache, the interconnect and DRAM
interface are not bottlenecks, and the machine uses a modern
scalable directory-based coherence protocol.

\begin{comment}

This view of scalability is unusual in that it doesn't involve varying
the number of cores executing a workload.  Traditional
scalability implies that a workload's throughput is $n$
times greater on $n$ cores than on a single core.  However, this
definition requires a way to construct executions for varying numbers of
cores, as well as a measure of throughput.  We instead look at a single
execution and consider whether the time to execute each operation is
affected by concurrent operations on other cores.  Intuitively, if the
execution time of each individual operation is the same as if it were
run in isolation, it means that the overall $n$-core system can execute
the operations at $n$ times the rate of a single core running those same
operations, which agrees with the traditional, informal definition of
scalability.

Conversely, when two cores on the abstract machine have read-write or
write-write sharing, data must move between cores.  As the number of
cores grows, the amount of concurrent data movement will also grow.
We define such communication to be non-scalable on our abstract machine
model.  Whether that is true on a real computer depends on whether each
write affects a growing number of reading cores, on how scalable the
computer's interconnect is, and on whether the cache coherence protocol
has serialization points such as directories that process coherence
messages at a limited rate.  Like our earlier assumption, we have
verified that real hardware generally agrees with this assumption.
%\S\ref{sec:eval:hardware} shows that scaling
%on real hardware agrees with the model in many regimes.

For example, in an implementation of \code{open} that acquires locks on
directories to check if the named file already exists,
\code{open("/tmp/a", O_CREAT|O_EXCL)} and
\code{open("/tmp/b", O_CREAT|O_EXCL)} will both write to \code{tmp}'s
lock and hence won't scale linearly on the abstract machine.  On the
other hand, in an alternate implementation that supports lock-free path
lookup and represents directories as hash tables with per-bucket locks,
these same calls will scale linearly on the abstract machine (barring
hash collisions) because they won't write to the same memory.
\XXX[AC][``Lock-free'' isn't the right term here because that still
permits writes.  Also, open could do other non-scalable things (like
allocate FDs), so maybe this needs to the worded to focus on the file
creation part.]

\XXX[AC][Set data structure example?  Hash set versus tree set?  Radix
set?  List set?]
\end{comment}


\subsection{System calls: stat}

%\XXX[AC][We use fstat instead of stat to focus on stating rather than
%path lookup.]

To get a sense of whether the commutativity rule matters for scalability
of real system interfaces, and how one might apply it in a constructive
fashion, we explore one specific example from the Linux system call API:
the \code{stat} system call.  The \code{stat} system call returns both the
length of the file and the number of hard links among other things.
Because the \code{stat}
system call returns the link count, it does not commute with
\code{link} and \code{unlink} operations on the same file, even if
the application is only interested in the file's length. This lack of
commutativity suggests that these operations cannot scale.

Indeed, in Linux, the \code{stat} system call does not scale when running
concurrently with \code{link} and \code{unlink}.
Several shared locks are acquired by \code{link} and \code{unlink}, and
it is difficult to restructure the code to isolate their effects.  But
how much of this non-scalability is due to potential implementation
defects in Linux, and how much of it is inherent due to the non-commutative
API?

To answer this question, we implemented a modified version of
\code{stat} in a preliminary version of \sys.  This
modified version lets applications specify which fields to return.  If
an application specifies all fields, it behaves like Linux's
\code{stat}.  However, if an application indicates that \code{stat}
should only return the file's size and not its link count, then,
unlike Linux's \code{stat}, that call to \code{stat} will commute with
\code{link} and \code{unlink}.

Using this modified \code{stat}, we run a benchmark that uses half of
the available cores to repeatedly \code{stat} a file, while the other
half of the available cores repeatedly \code{link} the file to a new,
unique name and then \code{unlink} the new name, causing the file's
link count to fluctuate.  Our implementation of \code{stat} is optimal
for this benchmark in that it moves at most one cache line: the cache
line containing the file's link count.

\begin{figure}
  \centering
  \input{graph/linkbench-xv6.tex}
  \caption{Throughput of \code{stat} calls, with $n/2$ cores calling
    \code{stat} and $n/2$ cores calling \code{link} and
    \code{unlink}.}
  \label{fig:linkbench}
\end{figure}

Figure~\ref{fig:linkbench} compares the results of two configurations
of this benchmark: one that calls \syscall{stat} in a non-commutative
way, requesting all file information including the link count, and one
that calls \syscall{stat} in a commutative way, requesting that the
link count be omitted from the result.  The difference is striking:
despite moving only a single cache line, the scalability of the
non-commutative configuration is severely limited, while the
commutative configuration scales perfectly.  The limited scalability
of the regular \syscall{stat} implementation is a direct consequence
of the interface-level non-commutativity of \syscall{stat} and
\syscall{link}/\syscall{unlink}.  Thus, we
believe that the proposed research will be important to real systems
and real APIs.

%Furthermore, this result indicates that our abstract
%machine model captures a key scalability property of real hardware.

