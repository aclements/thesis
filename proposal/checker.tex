\section{Checking the Scalable Commutativity Property}
\label{s:checker}

The Scalable Commutativity Property guides design, but for complex and
multi-layered software like an OS kernel with interacting subsystems, developers must
ensure that the implementation satisfies the property when expected.  One
question we would like to explore as part of this proposal's research is how
check whether an implementation obeys SCP.  As a starting point, we would like
to explore an approach based on logical variables and physical memory tracing.

At first glance it might seem easy to check if an operation scales linearly
on real hardware: run an application and the kernel under a cache profiler
and look for coherence misses.  However, without logical
commutativity, the cache profiler cannot distinguish between inevitable
coherence misses from operations that are not logically commutative, and
coherence misses that represent violations of scalable commutativity.  Hence,
while cache profiling is important for improving the performance of a system, we
need a different approach for focusing a developer's attention on the logically
commutative operations that don't scale linearly.

We can accomplish this with the help of a dynamic analysis tool that provides
logical variables to allow kernel developers to specify logical commutativity in
their source code.  The tool will use these annotations and physical memory tracing
to determine which operations scale linearly.  Combined with workload
generators designed to exercise the kernel, this will let us report pairs of
operations that violate the Scalable Commutativity Property, along with exactly
where the violation occurred and what memory was shared.

\subsection{Logical variables}
\label{sec:logicvar}

\XXX[rtm][should we call them ``logical objects'' instead? or abstract
objects? the word ``variable'' calls attention to the idea that they
will be modified -- but they aren't.]

We first introduce the notion of logical variables, a mechanism to help us
reason about logical commutativity.  A \textit{logical variable} is a unique
name for a logical part of the shared state, such as a virtual page in an
address space, that an operation either observes or modifies.  We name logical
variables using tuples, such as $\langle \mathrm{DirEnt}, \mathit{device},
\mathit{inode}, \mathit{name} \rangle$ for a directory entry, where
$\mathit{device}$ and $\mathit{inode}$ identify the directory and
$\mathit{name}$ is the file name within that directory.  The logical variables
of an operation can depend on its arguments as well as system state.  For
example, \code{open} ``reads'' a $\mathrm{DirEnt}$ logical variable for each
component in the pathname being opened, which requires mapping names to inode
numbers.  \code{creat} additionally ``writes'' the $\mathrm{DirEnt}$ of the
final component. Logical variables are intuitively related to the abstractions
application developers think about when using kernel subsystems.
\XXX[rtm][this last sentence needs more explanation. why is it true?
what about subsystems like the block cache, which deal with objects
private to the kernel, which the app writer may never think about?
or is the point that it's useful to be mindful of what the app writer
thinks when specifying logical variable -- but why? is the idea that
logical variables are only associated with the system call interface,
and never used deeper in the kernel?]

Logical variables make it easy to reason about when two operations logically
commute.  Two operations logically commute if they do not have any read-write
or write-write sharing at the level of logical variables.  That is, if
any logical variable written by one operation is either read or
written by the other, they do not logically commute; otherwise, they
do.
For example, two \code{creat}s of different names in the same directory
both read from the $\mathrm{DirEnt}$s leading to that directory, but
they write to different $\mathrm{DirEnt}$s and neither reads the
other's newly created $\mathrm{DirEnt}$, so they logically commute.

As another example, consider the VM system.  Non-overlapping
\code{mmap} operations commute and page faults commute with each
other, but a page fault does not commute with an \code{mmap} of the
faulting virtual page. We can
express this using logical variables for each virtual page, such as
$\langle \mathrm{VirtPage}, \mathit{asid}, \mathit{pagenum}\rangle$,
where $\mathit{asid}$ is some address space identifier and
$\mathit{pagenum}$ is the virtual page number.  \code{mmap} logically
writes each $\mathrm{VirtPage}$ in the range being mapped and page
fault reads from the $\mathrm{VirtPage}$ of the faulting virtual
address.

Logical variables allow the designer to think about each operation in
isolation and write down the logical variables it accesses.  Then, the designer
can think about logical commutativity of two operations by reasoning about the
logical read and write sets of those two operations.  This is
useful, because it allows thinking about subsystems without explicitly stating
the relationship of every pair of operations.  For instance, an
\code{mmap} of a file descriptor and a \code{close} of that file
descriptor do not commute, because closing a file descriptor would
cause a later \code{mmap} to fail.
However, a developer can think about the VM
system and file system independently as long as they use consistent
logical variable names; they don't have to consider this particular
pair of operations.

\subsection{Using logical variables}

To check a kernel, the developer must annotate the source code to denote logical
variables and operations to be checked by the checker; these operations
include systems calls as well as internal operations.  To collect logical and
physical traces, we plan to run an annotated kernel with applications on a
modified QEMU~\cite{qemu}, which we call \mtrace.  We will add support for
hypercalls so that software running within \mtrace can communicate events such
as logical variable declarations, call stack switches, and the range and type of
every memory allocation.  \mtrace also modifies the QEMU code generator to
intercept all loads and stores and can trace additional instructions such as
\code{call} and \code{ret} to produce call stacks.  This information is written
to a log file.

To identify logically commutative operations that don't scale, an
offline tool called \mscan analyzes the log recorded by \mtrace.  \mscan
collects every operation and attributes logical variable reads and writes as
well as physical memory reads and writes to operations.  It then tests every
pair of operations.  If there are no common logical variables written by one
operation and written or read by the other, then the operations logically
commute, and \mscan checks for physical memory that was written by one operation and
read or written by the other.  Any such overlap identifies operations
that execute sub-linearly and \mscan reports the names of the conflicting operations,
the call stacks of the accesses, and the address of the conflict, which it
resolves to a user-friendly structure field or array index using the type
information recorded for allocations and static variables.

