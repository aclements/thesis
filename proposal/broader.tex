\section{Broader Impact of Proposed Research}
\label{sec:broader}

We have a four-pronged approach for broader impact of our proposed
research: impact through advances in knowledge and discovery; impact
by outreach to under-represented minorities and women; impact by improved
education; and research community impact by dissemination of software
and traces.

\paragraph{Advances in knowledge and technology.} 

The long-term impact of this research is ideas about how to structure
software so that it is scalable by design, by thinking about interface
commutativity.  Our ideas about commutative interface design for
scalability apply equally well to other systems than operating systems,
and in the long term we hope that programmers find our ideas useful to
think about how to structure software to obtain good scalability.

In the more near term, we hope to contribute new techniques for how to
make operating systems more scalable.  Our goal is that these techniques
address problems that widely used operating systems experience, with
the hope that our solutions will influence those operating systems.
Because of our work with Linux~\cite{boyd-wickizer:scaling,clements:bonsai}, we
have developed a good understanding of those challenges as well as good
relationships with developers in the Linux community.  To make our results
easily adoptable we will release \sys publicly available as open source.

\paragraph{Outreach to under-represented minorities and women.}

Harvard's computer science department is a leader in involving women
in computer science.
%
The CS 50 introductory course, currently Harvard's 2nd largest class
in terms of undergraduate enrollment, has increased relative female
participation year over year (to 38\% women in 2012), which is
unusually high for such courses.
%
We will build on this outreach to involve women in the more advanced
classes offered as part of this research.
%
MIT EECS runs a summer mentorship program (the Women's Technology Program, or
WTP) for female high-school students.  WTP was set up to encourage high school
seniors to take up engineering---and particularly computer science---as a major
in their undergraduate school, and also to come to MIT for college.  Each female
student is assigned a faculty mentor, learns about a particular area, and works
on a short-term research project.  We will use the proposed work to excite these
students about operating systems.  We have found that the availability of xv6
(and its commentary) have made it easier for students with no background in
operating systems to get quick to the same level as students with much
background in C, Linux, etc.

\paragraph{Improved education.}

To reach out to educators and students all over the world we will make classes
and software available through MIT's Open Course Ware (OCW) and through MITx
(as part of edX). MIT's 6.828 course (operating system engineering), which
uses xv6, is one of the most visited courses at OCW and schools have
adopted it outside of MIT\@.  We will use this avenue, as well as MITx
courses, to popularize \sys.

\paragraph{Dissemination of software and traces.}

The PIs have historically made their software and services publicly available,
and have supported sizable user communities. We will make the source code to
the \sys implementations publicly available, as well as traces and
benchmarks. This way we can ensure reproducible scientific results while at the
same time providing data for future endeavors.

% LocalWords:  multicore PIs MSRA Kaashoek EECS mentorship WTP OCW JOS
% LocalWords:  adoptable
