% Systems programs, including operating systems kernels, network servers,
% and databases, are very hard to scale~\cite{barrelfish,linuxscalability}.
%
Systems software is often repeatedly
re-architected as increasing numbers of cores
turn innocuous constructs into bottlenecks.
%
This process can take tremendous effort over many years, as has been shown by
Linux and other systems.
%
A more attractive approach would be to build, \emph{once},
implementations that scale \emph{by design} to any number
of cores.
%
Unfortunately, scalability by design isn't always possible.
%
Some interfaces can always be made to scale;
%
others can scale only for some system states or for some
mixes of operations;
%
and others are always unscalable.
%
The difference is not always obvious and interface design decisions
that seem innocuous at the time can become serious bottlenecks down
the road.  For example, the requirement
that the Unix \syscall{open} system call return the \emph{minimum}
available file descriptor prevents it from scaling whenever multiple
threads in a process open files simultaneously.
%
Lacking tools for reasoning about interface scalability,
programmers instead measure implementation scalability.
%
This leads to a host of problems, including endless cycles of
benchmarking and rewriting, unclear performance contract between
software layers, fruitless searches for scalable
implementations, and incorrect assumptions that interfaces cannot
scale.
%
It is no wonder programmers resort to a reactive approach of waiting for
bottlenecks to appear.

The missing piece required for scalability by design is \emph{modular
scalability reasoning}.
%
Programmers need well-defined rules for analyzing scalability
independent of particular implementations.
%
The rules should distinguish situations where scalability is possible
from ones where it is not, so that programmers can focus their
attention.
%
Implementation effort is justified when the rules say scalability is
possible; re-design of interfaces or changes in interface usage
can avoid scalability limitations; and sometimes scalability is
inherently impossible, as when threads intentionally interact.
%
My thesis will propose and examine the \emph{external commutativity
  rule} as a foundation for modular scalability reasoning and explore
its applications to prominent software interfaces and the
implementation of multicore systems software.
