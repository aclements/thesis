\section{Collaboration Plan}
\label{sec:collabplan}

% \XXX[FK][Section must be title as above, and we get 2 extra pages
% (beyond the 15)]

To answer the research questions laid out in \S\ref{sec:research},
we have structured the work of this proposal in six tasks, as follows:

\begin{CompactItemize}

\item Formulating and proving the commutativity rule, and
  coming up with intuitive classes of commutativity.

\item Understanding the relationship between real hardware
  and our abstract machine, and what systems scale on one
  but not the other.

\item Applying the commutativity rule to the POSIX interface
  implemented by Linux.

\item Implementing a scalable OS kernel, \sys, based on a modified
  POSIX interface that makes more operations commute, and
  checking its implementation.

\item Building applications that use the commutativity rule. We plan
  to leverage our experience with Masstree~\cite{mao:masstree} to
  build a more scalable and complete key-value store or database.
  Commutativity rules will guide both the system calls used by the
  key-value store to interact with the OS, and the interfaces the
  store offers to its users. As part of this work, we will develop
  commutative mechanisms that achieve isolation with less scalability
  impact than fully general transactions can imply.

\item Developing tools for automatically or semi-automatically
  checking whether an interface is commutative.

\item Using our implementation of \sys and a related system, JOS, as teaching tools in
  classes on operating systems and multi-core software.

\end{CompactItemize}


Many of these tasks interact with one another.  To make progress, we
will explore these tasks in the context of individual sub-systems first,
and then apply them to an entire kernel or application.  In particular,
our yearly milestones are as follows:

\begin{CompactItemize}
\item {\bf Year 1.}
Formalize and prove the basic commutativity rule.
%
Investigate related rules. Find classes of commutative
interface that allow more general scalable implementations than the
specifically tailored implementation in Section~\ref{rulerule}.
%
Perform a case study of Linux system calls using a preliminary version
of the rule.  Choose one or two subsystems for improving scalability
through API changes.  (Two promising candidates are the virtual memory
system and the message-passing mechanisms such as sockets.)
Explore the performance of real hardware to understand important aspects
for the abstract machine.
Start designing a commutative key-value store interface.

\item {\bf Year 2.}
Classify Linux system calls using the commutativity classes developed
in year 1.
Design scalable interfaces for several POSIX subsystems chosen in year 1,
and build scalable implementations of these subsystems in \sys.
Explore situations where amortized scalability could be applied, such
as RCU garbage collection and load balancing.
Refine the rule based on our experience with the Linux case study,
and categorize potential commutativity classes.
Initial prototype of scalable key-value store on Linux.

\item {\bf Year 3.} 
Complete prototype of \sys.  Port key-value store to \sys and explore
its use of scalable \sys's system calls.  Formalize abstract
hardware model, commutativity classes, and models of amortized
scalability.  Develop an initial rule checker and apply it to the
subsystems developed in year 2.

\item {\bf Year 4.} 
Fine-tune and evaluate \sys and the key-value store application.
Use these prototypes in our classes, and make available for other
researchers and educators.  Complete the checker and apply it to the
entire kernel.  Extract commutativity and scalability specifications to
understand how to provide a contract for manpages.

\end{CompactItemize}

The PIs at Harvard and MIT already meet and collaborate frequently,
most recently on Masstree~\cite{mao:masstree}.  We plan to co-advise
Ph.D. students involved in this proposal and to continue meeting in
person weekly.  Although all of the PIs will be involved in all aspects
of this proposal, the focus at MIT will be on the \sys kernel, and the
focus at Harvard will be on the applications.

One major component of the proposed research is dissemination of knowledge
through public-domain code releases and development of course materials on
parallel programming.  We will jointly develop and maintain the code base
of our software infrastructure and foster a user community around work.
Our previous collaboration has been very successful.

