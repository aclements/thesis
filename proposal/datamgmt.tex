\newpage
\section*{Data Management Plan}

% Plans for data management and sharing of the products of
% research. Proposals must include a supplementary document of no more
% than two pages labeled “Data Management Plan”. This supplement should
% describe how the proposal will conform to NSF policy on the dissemination
% and sharing of research results (see AAG Chapter VI.D.4), and may include:
% 
% - the types of data, samples, physical collections, software,
%   curriculum materials, and other materials to be produced in the course
%   of the project;
% 
% - the standards to be used for data and metadata format and content
%   (where existing standards are absent or deemed inadequate, this should
%   be documented along with any proposed solutions or remedies);
% 
% - policies for access and sharing including provisions for appropriate
%   protection of privacy, confidentiality, security, intellectual property,
%   or other rights or requirements;
% 
% - policies and provisions for re-use, re-distribution, and the
%   production of derivatives; and
% 
% - plans for archiving data, samples, and other research products,
%   and for preservation of access to them.

The data generated through the work described in this proposal will consist of
papers, source code for the implementation of \sys and the key-value store,
benchmarks, and experimental results.  We will release the source code, data
from experimental results, and latex sources for papers in the public domain.

\paragraph{Data formats.}
All data generated will be labeled and stored in digital formats.  For papers,
we will keep both a PDF format of the document and the {\LaTeX} source.  For
documentation, we will store it in HTML format.  The source code we will store
it in public git repositories; project members will be able to check in code
into the repository, and anyone will be able to check out a copy of our
publicly accessible repositories.  Experimental results will be stored in
standard formats compatible with widely used software tools, such as
spreadsheets or ASCII text files.

\paragraph{Data retention.}
Data will be retained for at least three years beyond the award period, as
required by NSF guidelines.  The git version control system will ensure the
integrity of data, and simplify long-term retention.  All data will be retained
in secure network file storage by the individual research groups of the PIs, and
regularly backed up to prevent data loss.

\paragraph{Data sharing.}
The data will be made available for sharing upon official request within
12 months after conclusion of the award, or immediately after publication,
whichever occurs first.  Source code for our prototypes will be publicly
available through our git repositories.  All data shared will include
documentation, standards, and notations needed to interpret the data,
following commonly accepted practices in the field.  All published data
will be available for re-use and re-distribution.

\paragraph{Data privacy.}
This research project doesn't involve confidential data or source code.

\paragraph{Responsible data management.}
All members of the investigative team with data access will be
instructed and certified in responsible research and scholarship at
their respective institutions (e.g., the Responsible Conduct of Research
course provided by the MIT Office of Sponsored Projects).  All data
acquired and preserved in the context of this proposal will be further
governed by the policies of the Massachusetts Institute of Technology
and of Harvard University pertaining to intellectual property, record
retention, and data management.

