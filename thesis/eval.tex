\section{Performance evaluation}
\label{sec:eval}

The previous section showed that \fs and RadixVM achieve conflict-freedom
for nearly all commutative operations, which should result
in perfect scalability in
theory.  This section shows that these results translate to scalability on
real hardware for a complete operating system by answering the following
questions:

\begin{CompactItemize}

\item Do non-commutative operations limit performance on real
  hardware?

\item Do conflict-free implementations of commutative operations scale
  on real hardware?

\item Does optimizing for scalability sacrifice sequential performance?

\end{CompactItemize}


\subsection{Experimental setup}
\label{sec:topic:ben}

To answer these questions, we use \sys.
In addition to the operations analyzed in \S\ref{sec:model}, we scalably
implemented
other commutative operations (e.g., \code{posix_spawn})
and many of the modified POSIX APIs from
\S\ref{sec:posix}.
%
All told, \sys
totals \pyexpr{const["xv6-loc"]["all"]} lines of code, including user
space and library code.

We ran experiments on an 80-core machine with eight 2.4~GHz 10-core
Intel E7-8870 chips and 256~GB of RAM.  When varying the number of
cores, benchmarks enable whole sockets at a time, so each 30~MB
socket-level L3 cache is shared by exactly 10~enabled cores.
We also report single-core numbers for
comparison, though these are expected to be higher because one
core can use the entire 30~MB cache.

We run all benchmarks with the hardware prefetcher disabled because we
found that it often prefetched contended cache lines to cores that did
not ultimately access those cache lines, causing significant
variability in our benchmark results and hampering our efforts to
precisely control sharing.  We believe that, as large multicores and
highly parallel applications become more prevalent, prefetcher
heuristics will likewise evolve to avoid inducing this false sharing.

As a single core performance baseline, we compare against the same
benchmarks running on Linux~3.5.7 from Ubuntu Quantal.  Direct
comparison is difficult because
Linux implements many features \sys does not, but this comparison
indicates \sys's performance is sensible.
\XXX[AC][Different version from mtrace, which is a little awkward.]

\subsection{Microbenchmarks}
\label{sec:eval:microbenchmarks}

% Many interfaces provide stricter or more complex semantics than
% generally required by applications.  \code{fstat}, for example, combines
% several return values, including properties like a file's
% link count, while applications typically only need one or two of these
% properties, such as the file size or modification time.  Such interfaces
% often have limited commutativity, where slightly modified interfaces
% that better align with application needs may commute in more situations.
% \code{fstat} does not commute with \code{link} on the same inode because
% \code{link} modifies the link count that \code{fstat} returns, but if an
% application could request only, say, the file size, the resulting
% interface would commute with \code{link}.

% We call such cases, where operations do not commute according to the
% specification but weaker and more widely commutative semantics suffice
% for many applications, \emph{needless non-commutativity}.
% \XXX[AC][Avoid ``needless non-commutativity.''  The point of the
% commutative versions of the benchmarks is just to verify that it's the
% non-commutativity limiting scalability.]

% In this section, we explore the impact of non-commutative interfaces on
% scalability by examining the scalability of some of the needlessly
% non-commutative POSIX interfaces discussed in
% \S\ref{sec:posix}. \XXX[FK][I don't think we need this reference to
% section 6, and we could do without section 6, and turning it into a
% future work and discussion section.]

We evaluate scalability and performance on real hardware using two
microbenchmarks and an application-level benchmark.  Each benchmark has
two variants, one that uses standard, non-commutative POSIX APIs and
another that accomplishes the same task using the modified, more broadly
commutative APIs from \S\ref{sec:posix}.
%
By benchmarking the standard interfaces against
their commutative counterparts, we can isolate the cost of
non-commutativity and also examine the scalability of
conflict-free implementations of commutative operations.

We run each benchmark three times and report the mean.  Variance from
the mean is always under 4\% and typically under 1\%.  \XXX[AC][Except
mailbench with regular APIs on 80 cores, where something clearly went
bonkers.]

% We examine three different classes of
% non-commutativity: statbench exercises \code{fstat}, which combines
% several operations into one; openbench exercises file descriptor
% allocation, which has process-wide invariants; and sockbench exercises
% local sockets, which have strict ordering requirements.

\paragraph{statbench.} In general, it's difficult to argue that an
implementation of a
non-commutative interface achieves the best possible scalability for
that interface and that no implementation could scale better.  However,
in limited cases, we can do exactly this.  We start with statbench,
which measures the scalability of \code{fstat} with respect to
\code{link}.  This benchmark creates a single file that $n/2$ cores
repeatedly \code{fstat}. The other $n/2$ cores repeatedly
\code{link} this file to a new, unique file name, and then \code{unlink}
the new file name.  As discussed in \S\ref{sec:posix}, \code{fstat} does not
commute with \code{link} or \code{unlink} on the same file because
\code{fstat} returns the link count.  In practice,
applications rarely invoke \code{fstat} to get the link count, so \sys
introduces \code{fstatx}, which allows applications to request specific
fields (a similar system call has been proposed for
Linux~\cite{linux:xstat}).

\XXX[AC][Mention that commutative statbench helps verify our
disjoint-access parallelism assertion on real hardware.]

\newcounter{mysubfigure}[figure]
\renewcommand{\themysubfigure}{\thefigure(\alph{mysubfigure})}

\begin{figure*}
  % hbox prevents wrapping and fixes nasty vertical space when we
  % overfill
  \hbox{
    % Make subfigure labels refer to correct figure
    \stepcounter{figure}
    \input{graph/linkbench.tex}
    \refstepcounter{mysubfigure}
    \label{fig:linkbench}
    \hspace{-.22in}
    \input{graph/fdbench.tex}
    \refstepcounter{mysubfigure}
    \label{fig:fdbench}
    \hspace{-.22in}
    \input{graph/mailbench.tex}
    \refstepcounter{mysubfigure}
    \label{fig:mailbench}
    \addtocounter{figure}{-1}
  }
  \caption{Benchmark throughput in operations per second per core with
    varying core counts on \sys.
    The blue dots indicate single core Linux
    performance for comparison.}
\end{figure*}

% \begin{figure}
%   \centering
%   \input{graph/linkbench.tex}
%   \caption{\code{fstat} throughput with $n/2$ cores running \code{fstat}
%     and $n/2$ cores running \code{link}/\code{unlink}.}
%   \label{fig:linkbench}
% \end{figure}

We run statbench in two modes: one mode uses \code{fstat}, which does
not commute with the \code{link} and \code{unlink} operations performed
by the other threads, and the other mode uses \code{fstatx} to request
all fields except the link count, an operation that \emph{does} commute
with \code{link} and \code{unlink}.  We use a \refcache scalable
counter~\cite{clements:radixvm} for the link count so that the
\code{link}s and \code{unlink}s do not conflict, and place it on
its own cache line to avoid false sharing.
Figure~\ref{fig:linkbench} shows the results.  With the commutative
\code{fstatx}, statbench scales perfectly and experiences zero L2 cache
misses in \code{fstatx}, while \code{fstat} severely limits the
scalability of statbench.  

%This is no implementation fluke. 
To better
isolate the difference between \code{fstat} and \code{fstatx}, we run
statbench in a
third mode that uses \code{fstat}, but represents the link count
using a simple shared counter instead of \refcache.  In this mode, \code{fstat}
performs better (at the expense of \code{link} and \code{unlink}), but
still does not scale.  With a shared link count, each \code{fstat}
call experiences exactly one L2 cache miss (for the cache line
containing the link count), which means this is the most scalable that
\code{fstat} can possibly be in the presence of concurrent \code{link}s
and \code{unlink}s.  Yet, despite sharing only a single cache line, this
seemingly innocuous non-commutativity limits the
implementation's scalability.  One small tweak to make the operation
commute by omitting \code{st_nlink} eliminates the barrier to scaling,
demonstrating the cost of non-commutativity.

In the case of \code{fstat}, optimizing for scalability sacrifices some
sequential performance.  Tracking the link count with \refcache
(or some scalable counter) is necessary to make \code{link} and
\code{unlink} scale linearly, but requires \code{fstat} to reconcile the
distributed link count to return \code{st_nlink}.  The exact overhead
depends on the core count (which determines the number of \refcache
caches), but with 80~cores, \code{fstat} is \pyexpr{times(
  mean(linkbench.filter(("iv.st_nlink",True),("iv.stats",1),("iv.FS_NLINK_REFCOUNT","refcache::"))["dv.cycles/stat"]),
  mean(linkbench.filter(("iv.st_nlink",True),("iv.stats",1),("iv.kernel","Linux"))["dv.cycles/stat"]))}
more expensive than on Linux.  \XXX[AC][Haven't tried optimizing this.]
In contrast, \code{fstatx} can avoid this overhead unless link counts are
requested; like \code{fstat} with a shared count, it performs similarly to Linux's \code{fstat} on
a single core.
%\pyexpr{int(mean(linkbench.filter(("iv.st_nlink",True),("iv.stats",1),("iv.FS_NLINK_REFCOUNT","::"))["dv.cycles/stat"]))}~cycles
% Above 20~cores, the shared link count and \refcache link count
% implementations of \code{fstat} scale similarly.

\paragraph{openbench.} Figure~\ref{fig:fdbench} shows the results
of openbench, which stresses the file descriptor allocation performed by
\code{open}.  In openbench, $n$ threads concurrently \code{open} and
\code{close} per-thread files.  These calls do not commute because each
\code{open} must allocate the lowest unused file
descriptor in the process.  For many applications, it suffices to return
any unused file descriptor (in which case the \code{open} calls commute),
so \sys adds an \code{O_ANYFD} flag to \code{open}, which it implements
using per-core partitions of the FD space.  Much like
statbench, the standard, non-commutative \code{open} interface limits
openbench's scalability, while openbench with \code{O_ANYFD} scales
linearly.  Furthermore, there appears to be no performance penalty to
\fs's \code{open}, with or without \code{O_ANYFD}: at one core, both
cases perform identically and outperform Linux's \code{open} by
\pyexpr{percent(
  mean(fdbench.filter(("iv.cores",1),("iv.kernel","xv6"),("iv.any_fd",False))["dv.opens/sec"])
  /mean(fdbench.filter(("iv.cores",1),("iv.kernel","Linux"))["dv.opens/sec"])
  -1)}.
Some of the performance difference is because \sys doesn't implement things like
permissions checking, but much of Linux's overhead comes from locking
that \fs avoids.

% \begin{figure}
%   \centering
%   \input{graph/fdbench.tex}
%   \caption{\code{open} throughput with varying FD allocation policies.}
%   \label{fig:fdbench}
% \end{figure}

\begin{comment}
\paragraph{sockbench.}

Figure~\ref{fig:sockbench} shows the results of sockbench, which
stresses local sockets.  In sockbench $n$ client processes repeatedly
send a 1-byte message over a local socket to $n$ server processes and
wait for a 1-byte response. The clients send the 1 byte message over a
datagram socket that is shared among all servers.  The POSIX API doesn't
require that datagram messages be delivered in order but most operating
systems do enforce this ordering because a single queue for a socket is
the most straightforward implementation.  This unnecessary ordering,
however, makes \code{sendto} invocations on a datagram socket needlessly
non-commutative.

\sys allows for out of order delivery of datagram messages, making
invocations to \code{sendto} commute with each other.  \sys takes
advantage of this commutativity to achieve scalability as follows. When
\sys notices that several cores are receiving from a shared datagram
socket, it partitions the socket among the cores.  A sending core puts a
message into its local partition of the socket, unless its partition is
full.  If the partition is full, \sys invokes a scalable load balancing
algorithm to deliver the message into another partition.

As the results in Figure~\ref{fig:sockbench} show, \sys scales
perfectly, because each pair of a client and a server communicate
through its local partition of the datagram socket.  Thus, cores don't
need to share any cache lines, and \sys scales perfectly.  Linux scales
until a small number of cores, and then collapses because of a contended
spin lock protecting the in-kernel queue for the datagram socket.
\sys's scalability doesn't come at the cost of single-core performance;
in fact, \sys's single core performance is better than Linux.

If an application performs its own load balancing, it could implement
\sys's approach at the application level by setting up $n$ datagram
sockets, for each pair of a client and a server.  With this setup
invocations to \code{sendto} commute because they involve different
datagram sockets.  Unfortunately, today this setup does not result in
better scalability on Linux.  The line labeled ``$n$ datagram sockets''
shows better scalability than the line ``1 datagram socket'', but it
also collapses. At 20 cores, a spin lock protecting the name lookup in
\code{sendto} becomes contended.

The application could perform the name lookup once by a setting up a
stream socket at the beginning, and then using \code{send} to
communicate.  This setup results in better scalability (see the line
labeled ``Linux with $n$ streams''). But, in this setup, a spin
lock in the scheduler becomes a bottleneck (\XXX[FK][double check]).
Clearly, Linux developers could remove these bottlenecks.  What is nice
about our approach is that the commutativity rule makes clear that these
locks can be removed and that \tool{} can catch these non-scalable
invocations that should be scalable.  \XXX[FK][We should mention
somewhere that we have a model for unordered and ordered sockets.]

\begin{figure}
  \centering
  \input{graph/usocket.tex}
  \caption{Scalability of $n$ clients concurrently sending and receiving 1
    byte messages to/from $n$ server  processes.}
  \label{fig:sockbench}
\end{figure}
\end{comment}

%     lazy unmap [doing more than one thing]
%     thread-level mmap?
%     stat vs. fstat (name lookup)

\subsection{Application performance}
\label{sec:eval:app}

\XXX[FK/AC][Cite RadixVM on importance of scalable VM for application
performance. This may be unnecessary given what section 6.2 already says.]

Finally, we perform a similar experiment using a simple mail server to
produce a system call workload more representative of a real
application.  Our mail server uses a sequence of separate, communicating
processes, each with a specific task, roughly like qmail~\cite{qmail}.
{mail-enqueue} takes a mailbox name and a single message on
stdin, writes the message and the envelope to two files in a
mail queue directory, and notifies the queue manager by writing the
envelope file
name to a Unix domain datagram socket.  {mail-qman} is a long-lived
multithreaded process where each thread reads from the notification
socket, reads the envelope information, opens the queued message, spawns
and waits for the delivery process, and then deletes the queued message.
Finally, {mail-deliver} takes a mailbox name and a single message
on stdin and delivers the message to the appropriate Maildir.
The benchmark models a mail client with $n$ threads that continuously
deliver email by spawning and feeding {mail-enqueue}.

As in the microbenchmarks, we run the mail server in two configurations:
in one we use lowest FD, an order-preserving socket for queue
notifications, and \code{fork}/\code{exec} to spawn helper processes; in
the other we use \code{O_ANYFD}, an unordered notification socket, and
\code{posix_spawn}, all as described in \S\ref{sec:posix}.
%\code{openbench} explored the limitations of lowest
%FD.
For queue notifications, we use a Unix domain datagram socket;
\sys implements this with a single shared queue in ordered mode and with
per-core message queues with scalable load balancing in unordered
mode.  Finally, because \code{fork} commutes with essentially no other
operations in the same process,
\sys implements \code{posix_spawn} by constructing the new process image
directly and building the new file
table. This implementation is conflict-free with most other operations,
including operations on
\code{O_CLOEXEC} files (except those specifically \code{dup}ed into the
new process).

Figure~\ref{fig:mailbench} shows the resulting scalability of these two
configurations.  Even though the mail server performs a much broader mix
of operations than the microbenchmarks and doesn't focus solely on
non-commutative operations, the results are quite similar.
Non-commutative operations cause the benchmark's throughput to collapse
at a small number of cores, while the configuration that uses
commutative APIs achieves \pyexpr{times(
  mean(mailbench.filter(("iv.cores",80),("iv.alt","all"),("iv.kernel","xv6"))["dv.messages/sec"]),
  mean(mailbench.filter(("iv.cores",10),("iv.alt","all"),("iv.kernel","xv6"))["dv.messages/sec"]))}
scalability from 1~socket
(10~cores) to 8~sockets.
%
% In the non-commutative case, mailbench on \sys slightly outperforms
% Linux.
%
\XXX[AC][We could also make the point that, the commutative APIs
outperform the non-commutative APIs on one core, in addition to
out-scaling them.]

\XXX[AC][In the commutative case, our scalability is currently limited
by the growing chains in the fixed-size dcache hash table of the
Maildir.  A resizable hash table would probably help.]

\XXX[AC][In the non-commutative case, the top problem is fork by far.
However, with \code{posix_spawn} instead of fork, we spend all of our
time in \code{posix_spawn}, which scales but is slow.  If its
\emph{sequential} performance were better, the other things mailbench
does would matter more.]

% \begin{figure}
%   \centering
%   \input{graph/mailbench.tex}
%   \caption{Throughput of $n$ mail agents sending messages to a mail
%     server, which delivers them to a Maildir.}
%   \label{fig:mailbench}
% \end{figure}

\begin{comment}
\begin{figure}
  \centering
  \input{graph/forktest.tex}
  \caption{Scalability of $n$ cores forking and exiting a process on xv6
    and Linux.}
  \label{fig:forktest}
\end{figure}
\end{comment}
