% -*- fill-column: 80 -*-

\section{Related work}
\label{sec:related}

% The primary contribution of this paper is the commutativity rule, its use in the
% design and implementation of operating systems, and a new tool called \tool
% that can test if
% an implementation meets the commutativity rule.
% This section relates our
% contributions to previous work on thinking about scalability, on designing
% parallel operating systems, on exploiting commutativity, and on automated
% checking.

The scalable commutativity rule is to the best of our knowledge the
first observation to directly connect scalability to interface
commutativity.
% Further, to make the rule applicable to real-world
% interfaces, we define a novel form of commutativity: \SRI
% commutativity.
This section relates the rule and its use
in \sys and \tool to prior work.

\subsection{Thinking about scalability}

Israeli and Rappoport introduce the notion of disjoint-access-parallel
memory systems~\cite{israeli:disjoint-access}.  Roughly, if a shared memory system
is disjoint-access-parallel and a set of processes access disjoint memory
locations, then those processes scale linearly.  Like the commutativity rule,
this is a conditional scalability guarantee: if the application uses shared
memory in a particular way, then the shared memory implementation will scale.
However, where disjoint-access parallelism is specialized to the memory system
interface, our work encompasses any software interface.
% However, where disjoint-access parallelism is specialized to the memory system
% interface and can be demonstrated with specific implementations, the
% commutativity rule applies to any software interface and is thus a statement
% about the existence of an implementation.
%
Attiya et al. extend Israeli and Rappoport's definition to additionally require
conflict-free operations to scale~\cite{attiya:disjoint}.
%
Our work builds on the assumption that
memory systems behave this way, and we indirectly confirm that real hardware
closely approximates this behavior~(\S\ref{sec:eval}).

% They have a fixed interface (asynchronous shared memory) and show specific
% implementations with disjoint-access parallelism.

% We borrow their operation-centric notion of scalability: rather than focusing
% on whether a workload performs $n$ times better on $n$ cores, they/we focus on
% whether two operations execute more slowly when executed concurrently.

% Faced with a similar problem that they have control over what's below the
% interface, but not what's using it.  So they a similar sort of conditional
% guarantee: if the application uses the interface in this particular way, the
% implementation of the interface will perform this way.

% Disjoint-access parallelism is the architectural mirror of the commutativity
% rule.

% Whereas DAP is concerned with the shared memory interface and down, we assume
% DAP and work from the memory interface up.

Both the original disjoint-access parallelism paper and subsequent work,
including the paper by Roy et al.~\cite{roy:limits-dap}, explore the scalability
of processes that have some amount of non-disjoint sharing, such as
compare-and-swap instructions on a shared cache line or a shared lock.
Our work takes a black-and-white view because we have found that, on real
hardware, a single modified
shared cache line can wreck
scalability (\S\ref{sec:eval}).

The Laws of Order~\cite{law:orders} explore the relationship between an
interface's \emph{strong non-commutativity} and whether its implementation
requires atomic instructions or fences (e.g., \code{mfence} on the x86) for
correct concurrent execution.  These instructions slow down execution by
interfering with out-of-order execution, even if there are no memory access
conflicts.  The Laws of Order resemble the commutativity rule, but draw
conclusions about sequential performance, rather than scalability.
\XXX[FK][It would be great to have an example to nail this point. I am thinking
but along of the lines of per-core data structures, which need per-core locks
(and thus mfence instructions), but allow scalability.  I would like to drive
the point that mfence instructions are orthogonal to scalability. does that make sense?]
%
Paul McKenney
explores the Laws of Order in the context of the Linux kernel, and points out
that the Laws of Order may not apply if linearizability is not
required~\cite{lwn:law}.

% A simplifying interpretation of the Laws of Order and the commutativity rule is
% that the former states when operations cannot be implemented efficiently
% on each individual processor (assuming that expensive synchronization
% instructions limit performance), while the latter states when operations can
% be implemented in a manner that scales when adding more processors (under the
% assumption of disjoint-access parallelism).  An interesting area of future
% research is exploring the
% assumptions underlying each observation and exploring interfaces for which
% neither observation applies.

% \XXX[AC][If we want to say something here, contrast lock-/wait-freedom against
% conflict-freedom.]
% Theoreticians generally analyze implementations of concurrent
% data structures in terms of properties like lock-freedom and
% wait-freedom~\cite{herlihy:art}, and such data structures can help
% realize scalable implementations of commutative interfaces.
% However, while these notions guarantee progress, they do not imply that a data
% structure will perform or scale well.

It is well understood that cache-line
contention can result in bad scalability. A clear example is the design of
the MCS lock~\cite{MCS}, which eliminates scalability collapse by avoiding
contention for a particular cache line.  Other good examples include
scalable reference counters~\cite{approx:counter,snzi:podc}.
%
The commutativity rule builds on this understanding and identifies when
arbitrary interfaces can avoid conflicting memory accesses.

\subsection{Designing scalable operating systems}

Practitioners often follow an iterative process to
improve scalability: design, implement, measure, 
repeat~\cite{cacm-real-world}.
%
Through a great deal of effort,
this approach has led kernels such as Linux to scale well
for many important workloads. However, Linux still has many
scalability bottlenecks, and absent a method for reasoning about
interface-level scalability, it is unclear which of the bottlenecks
are inherent to its system call interface.  This paper identifies situations
where POSIX permits or limits scalability and points out specific interface
modifications that would permit greater implementation scalability.

Multikernels for multicore processors aim for scalability by avoiding shared
data structures in the kernel~\cite{barrelfish:sosp,wentzlaff:fos}.  These
systems implement shared abstractions using distributed systems techniques (such
as name caches and state replication) on top of message passing.  It should be
possible to generalize the commutativity rule to distributed systems, and relate
the interface exposed by a shared abstraction to its scalability, even if
implemented using message passing.

The designers of the Corey operating system~\cite{boyd-wickizer:corey} argue for
putting the application in control of managing the cost of sharing without
providing a guideline for how applications should do so;
the commutativity rule could be a helpful guideline
for application developers.

\subsection{Commutativity}

% Weihl and Steele's definition are almost identical, though Steele clearly says
% that the bytes in memory must be the same regardless of order and it's not
% clear if Weihl is talking about an "abstract" state.  Weihl comes at it from a
% DB perspective and has a lot to say about transactions and recovery.

The use of commutativity to increase concurrency has been widely explored.
%
Steele describes a parallel programming discipline in which all operations must
be either
causally related or commutative~\cite{steele:commutativity}.  His work
approximates commutativity as conflict-freedom.  Our work shows that commutative
operations always have a conflict-free implementation, making Steele's model
more broadly applicable.
%
% \XXX[AC][This one seems only tangentially related:] Herlihy and Wing use
% commutativity to prove the correctness of a concurrent queue in their paper
% introducing linearizability~\cite{herlihy:linearizability}.
%
Rinard and
Diniz describe how to exploit commutativity to automatically parallelize
code~\cite{rinard:commutativity}.  They allow memory conflicts, but generate
synchronization code to ensure atomicity of commutative operations.
%
Similarly, Prabhu et al.\ describe how to automatically parallelize code using
manual annotations rather than automatic commutativity
analysis~\cite{prabhu:cset}.
%
Rinard and Prabhu's work focuses
on the \emph{safety} of executing commutative operations concurrently.  This
gives operations the opportunity to scale, but does not ensure that they will.
Our work focuses on scalability: given concurrent, commutative
operations, we show they have a scalable implementation.

The database community has long used logical readsets and writesets, conflicts,
and execution histories to reason about how transactions can be interleaved
while maintaining serializability~\cite{bernstein:concurrency}.
%
Weihl extends this work to abstract data types by deriving lock conflict
relations from operation commutativity~\cite{weihl:commutativity}.
Transactional boosting applies similar techniques in the context of software
transactional memory~\cite{herlihy:boosting}.
%
Shapiro et al.\ extend this to a distributed setting, leveraging commutative
operations in the design of replicated data types that support updates during
faults and network partitions~\cite{shapiro:conflict-free,shapiro:convergent}.
%
Like Rinard and Prabhu's work, the work in databases and its extensions
focuses on the \emph{safety} of executing commutative operations concurrently,
not directly on scalability.

% \XXX[AC][I think we beat this to death better in other places.]
% Definitions of commutativity are nearly as diverse as the work that uses
% commutativity.  However, these definitions commonly require that operations
% commute in all states.  Our definition of commutativity is uniquely
% state-sensitive and therefore suitable for highly stateful interfaces such as
% POSIX.

\subsection{Test case generation}

Prior work on concolic testing~\cite{godefroid:dart,sen:cute}
and symbolic execution~\cite{cadar:exe,cadar:klee} generates test
cases by symbolically executing a specific implementation.
%
Our \tool tool uses a combination of symbolic and concolic execution, but
generates
test cases for an {\em arbitrary} implementation based on a model
of that implementation's interface.  This resembles
QuickCheck's~\cite{claessen:quickcheck} or Gast's~\cite{koopman:gast}
model-based testing, but uses symbolic techniques.
%
% AC: Misleading; \tool *does* want to achieve path coverage
% Unlike prior work, \tool uses symbolic execution to extract precise
% commutativity conditions from a model rather than to achieve path
% coverage in an implementation.
%
% As we discuss in \S\ref{sec:model:generator}, \tool
% generates test cases by enumerating all possible (non-isomorphic)
% assignments that satisfy the commutativity condition, instead of
% generating one test case per path.
Furthermore, while symbolic execution systems
often avoid reasoning precisely about symbolic memory accesses (e.g.,
accessing a symbolic offset in an array),
\tool's test case generation aims to achieve \emph{conflict} coverage
(\S\ref{sec:model:generator}), which tests different access
patterns when using symbolic addresses or indexes.

%%  LocalWords:  commutativity scalability Rappoport linearizable et al mfence
%%  LocalWords:  McKenney linearizability scalable MCS POSIX Multikernels Weihl
%%  LocalWords:  multicore multikernel readsets writesets serializability Diniz
%%  LocalWords:  Herlihy Rinard Prabhu Transactional transactional Steele's
%%  LocalWords:  QuickCheck Gast
