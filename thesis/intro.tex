\section{Introduction}
\label{sec:intro}
\label{intro}

\cbstart The state of the art for evaluating the scalability of multicore software
is to choose a workload, plot performance at varying
numbers of cores, and use tools such as differential
profiling~\cite{mckenney:differential} to
identify scalability bottlenecks.  \XXX[AC][Should we cite anything
for this?  Is there anything to cite besides our own papers?]
%
This focuses developer effort on real issues, but 
has several drawbacks.
%
Different workloads or higher core counts often exhibit new bottlenecks.
%
It's unclear which bottlenecks are fundamental, so developers may
give up without realizing that
a scalable solution is possible.
%
Finally, this process happens so late in the development process
that design-level solutions such as improved interfaces may be
impractical.

This paper presents a new approach to scalability that
starts at a higher level: the software interface.
%
This makes reasoning about scalability possible before an
implementation exists and before the necessary hardware is available
to measure the implementation's scalability.  It can highlight
inherent scalability problems, leading to alternate interface designs.
And it sets a clear scaling target for the implementation of a
scalable interface.

Scalability is often considered an implementation property, not an
interface property, not least because what scales depends on hardware.
%
However, if we assume a
shared-memory multicore processor with caches kept
coherent by a MESI-like protocol~\cite{papamarcos:mesi},
general scalability arguments are possible.
%
On such processors, a core can scalably read and write data
it has cached exclusively, and scalably read data it has cached
in shared mode. Writing a cache line that was last
read or written by another core is not scalable, however,
since the coherence protocol serializes ownership
changes for each cache line, and because the shared
interconnect may serialize unrelated transfers~\cite[\S4.3]{boyd-wickizer:thesis}.
%
% rtm: the reason for lack of scaling is the serialization,
% not the absolute amount of time taken.
%
% since that requires transferring ownership, which
% can take thousands of cycles and slow down concurrent
% cache line transfers~\cite[\S4.3]{boyd-wickizer:thesis}.
% In other words, modern multicore hardware is roughly disjoint-access
% parallel~\cite{israeli:disjoint-access,attiya:disjoint}.

We therefore say that a set of operations scales if their implementations
have \emph{conflict-free} memory accesses, where no core writes a
cache line that was read or written by another core.
%
When memory accesses are conflict-free, adding more cores will produce
a linear increase in capacity.
%Such
%operations do not experience coherence misses and do not slow each
%other down. 
% E- Can we get away w/o that sentence?
This is not a perfect model of the complex realities of
modern hardware, but it is a good approximation. 
% \XXX[AC][Make it
%clear that, informally, we equate conflict-free with scalable, but
%formally always state it in terms of conflict-freedom.]

At the core of our approach is this \emph{scalable commutativity rule}: In any
situation where several operations \emph{commute}---meaning
there's no way to distinguish their execution order using the
interface---they have an implementation whose memory accesses
are conflict-free during those operations.
Or, more concisely,
\textbf{whenever interface operations commute, they can be implemented in a way
  that scales.}
%
%Commutativity is an interface property:
%two operations commute if their order of execution does not affect
%their results, and the results of later operations cannot distinguish
%which operation executed first.
%
%\cbstart
%Scalability is an implementation property:
%two operations \emph{scale} if concurrent
%invocations do not slow each other down, so additional cores produce a
%linear increase in capacity.
% Because commutativity considers only the externally visible
% behavior of an interface, it is a property of the interface
% specification that is
% independent of its implementation.
%In other words, any interface has
%a certain inherent scalability that an implementation may
%exploit.
%\XXX[rtm][``a certain inherent scalability''
%sounds like we're talking about different amounts of scalability,
%rather than a yes-or-no property. can we say a few words to
%introduce what the amounts are? or maybe we want ``In other words,
%in situations where interface calls commute, scalable implementations
%are possible.'' though that is just a repeat of the rule,
%so maybe we should just omit this sentence.]

\cbstart
Connections between commutativity and concurrency are well
established in the literature.  Previous work, however, has focused on
using commutativity to reason about the safety of executing operations
concurrently (see \S\ref{sec:related}).  Our work is complementary: we
use commutativity to reason about scalability.
\cbend


% Hence, stated more precisely, the rule says that \emph{whenever two
%   operations commute, they have a conflict-free implementation.}
The commutativity rule makes
intuitive sense: when operations commute, their
results (return value and effect on system state) are independent of order.
%operation happens before or after the writes performed by the other
%operation. 
Hence, communication between commutative operations is
unnecessary, and eliminating it yields conflict-free implementations.
%
This intuitive version of the rule is useful in practice, but
not precise enough to reason
about formally.
%
\S\ref{sec:rule} formally defines the commutativity rule and proves
the correctness of the formalized rule.

An important consequence of this presentation is a novel form of
commutativity we call \emph{\SRI\ commutativity}.
%
The usual definition of commutativity (e.g., for algebraic operations)
is so stringent that it rarely applies to the complex, stateful
interfaces common in systems software.
%
\SRI\ commutativity, in contrast, is \emph{state-dependent} and
\emph{interface-based}, as well as \emph{monotonic}.
%
%Rather than applying to operations in isolation, \SRI\ commutativity
%applies to regions of specific executions.
%
When operations commute in the context of a specific system state,
specific operation arguments, and specific concurrent operations, we
show that an implementation exists that is conflict-free \emph{for that state
  and those arguments and concurrent operations}.
%
This exposes many more opportunities to apply the rule to real
interfaces---and thus discover scalable implementations---than a more
conventional notion of commutativity would.
%
Despite its logical state dependence, \SRI\ commutativity is
interface-based: rather than requiring all operation orders to produce
identical internal states, it requires the resulting states to be
indistinguishable via the interface.
%
\SRI\ commutativity is thus independent of any specific
implementation, enabling developers to apply the rule directly to
interface design.


% \XXX[AC][The above text doesn't explain why it's region-oriented (which
% is different from state-dependent!)  OTOH, region-oriented-ness isn't
% really that interesting.  Maybe we should drop it from the acronym and
% just discuss it in passing when we formalize commutativity?]

\cbstart The commutativity rule leads to a new way to design scalable
software:
%
first, analyze the interface's commutativity, and then design an
implementation that
scales in commutative situations. \cbend
%\XXX[rtm][to what does ``interface'' refer? to the specification of
%a single operation? what does it mean for a single operation to
%commute? commute with what? or does interface here refer to a
%set of operations, as in ``the POSIX interface?''
%maybe we want ``first, determine which pairs of an interface's operations
%commute, and then implement each operation so that it
%is conflict-free with the implementation of each other operation
%with which it commutes.'']
% E- Not sure that applies still.
%
For example,
consider file creation in a POSIX-like file system. Imagine that
multiple processes create files in the same directory at the same
time. Can the creation system calls be made to scale? Our
first answer was ``obviously not'': the system calls modify the same
directory, so surely the implementation must
serialize access to the directory. But it turns
out these operations commute if the two files have different names
(and no hard or symbolic links are involved) and, therefore, have an
implementation that scales for such names.
One such implementation represents each directory as a hash table
indexed by file name, with an independent lock per bucket,
so that creation of differently named files is conflict-free, barring
hash collisions.
%
Before the rule, we tried to determine if these
operations could scale by analyzing all of the \emph{implementations}
we could think
of.  This process was difficult, unguided, and itself did not scale to
complex interfaces, which
motivated our goal of reasoning about
scalability in terms of interfaces.

\begin{comment}
Applying the rule to real-world software systems poses two challenges.
%
First, the usual definition of commutativity is limited in ways that
restricts its applicability to complex, stateful interfaces.  To
address this challenge, \S\ref{sec:rule} defines a novel form of
commutativity named \emph{\SRI\ commutativity}.
%
\SRI commutativity is \emph{state-dependent}: rather than requiring
operations to be order-invariant in all states, \SRI\ commutativity
allows developers to reason not only about specific arguments for
which operations commute, but also in which system states they
commute, exposing more opportunities to apply the rule.  The above
example from POSIX shows why this is important: creating files with
different names does \emph{not} commute if the names are symlinks to
the same name, but this one case does not limit the rule's
applicability to other file system states.
%
\SRI\ commutativity is \emph{region-oriented}: rather than applying to just
two operations, \SRI\ commutativity applies to a region of an execution
trace, which may contain several operations, again exposing more
opportunities to apply the rule.
%
Finally, it is \emph{interface-based}: rather than requiring all
operation orders to produce identical internal states, it requires the
resulting states to be indistinguishable via the interface, allowing
developers to reason about commutativity independent of any specific
implementation.

\XXX[AC][Spread the above new terminology to the rest of the paper.]
\end{comment}

Complex interfaces can make it difficult to spot and reason about all
commutative cases even given the rule.  To address this challenge,
\S\ref{sec:model} introduces a tool named \tool that automates this
reasoning.  \tool takes an interface model
in the form of a simplified, symbolic implementation, computes precise
conditions under which sets of operations commute, and tests an
implementation for conflict-freedom under these conditions.  This tool can be
integrated into the development process to drive initial design and
implementation, to incrementally improve existing implementations, or to
help developers understand the commutativity of an interface.
%
%\XXX[FK][Can we
%working the following point into this paragraph? Before we had \tool, we
%evaluated the scalability of a system by choosing a workload, running it on
%some
%hardware, and plotting its performance at varying numbers of cores.]
% E- No we can't because we make that point in para1.
\cbend

This paper demonstrates the value of the commutativity rule and \tool
in two ways.
In \S\ref{sec:posix}, we explore the commutativity of POSIX
and use this
understanding both to suggest 
guidelines for designing interfaces
whose operations commute and
to propose specific modifications to POSIX
that would allow for greater scalability.

In \S\ref{sec:fs}, we apply \tool to a simplified model of
\pyexpr{len(mscan.calls)} POSIX file system and virtual memory system
calls.
%
From this model,
\tool generates \pyexpr{mscan.ntestcases} tests of commutative
system call pairs, all of which can be made conflict-free
according to the rule.  We use these tests to
guide the implementation of a new research operating system kernel
named \sys.  \sys has a novel virtual memory system
(RadixVM~\cite{clements:radixvm}) and in-memory file
system (named \fs).  \tool determines that \sys is conflict-free for
\pyexpr{mscan.ntestcases - mscan.xv6.shared} of the
\pyexpr{mscan.ntestcases} tests, while Linux is conflict-free
for \pyexpr{mscan.ntestcases - mscan.linux.shared}
tests.  Some of the commutative cases where Linux doesn't scale are important
to applications, such as commutative
\code{mmap}s and creating different files in a shared directory.
\S\ref{sec:eval} confirms that commutative conflict-free system calls
translate to better application scalability on an 80-core machine.


% \cbstart
% We demonstrate that
% {\tool} finds many non-obvious situations where operations commute; that
% {\tool} finds useful opportunities to increase implementation
% scalability; and that the test cases that {\tool} generates are a useful way
% to drive the development of an implementation by pinpointing
% cases in which the implementation could scale, but doesn't.


\begin{comment}
The contributions of the paper are as follows:
\begin{CompactItemize}

\item The commutativity rule;

\item The \tool{} tool;

\item \fs: a case study of using \tool to design a scalable file system;

\item A scalability evaluation with microbenchmarks and one application on an
  80-core machine.

\end{CompactItemize}
\end{comment}

% The paper is organized as follows.  \S\ref{sec:related} relates our thinking
% about scalability to previous work. \S\ref{sec:rule} introduces the
% commutativity rule and argues its correctness. \S\ref{sec:posix} applies the
% rule to an OS interface, identifying opportunities for
% commutativity. \S\ref{sec:model} describes \tool{}.  \S\ref{sec:fs} describes
% \fs and how \tool{} drove its implementation. \S\ref{sec:eval} shows that using
% the commutativity rule results in better scalability on real
% hardware. \S\ref{sec:concl} summarizes our conclusions.

% LocalWords:  rule's RCU
