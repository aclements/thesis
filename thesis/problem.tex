An example will show why commutativity rules can help even experienced
programmers understand the inherent scalability of an interface.
%
Consider the \syscall{open} system call, as used to create a file in a
directory.
%
Further imagine that two \syscall{open} calls are used to create two
files in the same directory at the same time.
%
Can those system calls be made to scale?
%
Our first answer was ``obviously not.''
%
The two system calls modify the same shared state, namely the containing
directory.
%
Surely in any implementation the two calls would contend on this shared
state, either via a lock or via non-blocking synchronization on one or
more shared cache lines. (Both forms of contention limit scalability.)
%
But commutativity shows this isn't really true.
%
If the two system calls create files \emph{with different names}, then the
success or failure of each system call is independent; later system calls can't
distinguish which completed first; and the operations commute.
%
Our commutativity rules say, therefore, that there must exist some
implementation of \syscall{open} that scales for these calls.
%
In hindsight, this implementation is straightforward.
%
If the directory is implemented as a hash table, and each hash bucket has an
independent lock, the two file creations can proceed (most likely) without
communicating.
%
The authors initially believed that this example could not scale; our
mistake was to analyze only the implementations we could think of for
scalability. It was examples like this that motivated our search for
the rule.
%
And file creation is not the only example: commutativity rules have pointed us
to similar insights in other interfaces, including the virtual memory system and
\syscall{mmap}.


