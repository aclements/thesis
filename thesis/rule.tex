% -*- fill-column: 72 -*-

\section{The scalable commutativity rule}
\label{sec:rule}

This section addresses two questions: What is the precise
definition of the scalable commutativity rule, and why is the rule true?
%
We answer these questions using a formalism based on abstract actions, histories,
and implementations.
%
The formalism relies on \SIM commutativity, whose generality broadens
the rule's applicability to complex software interfaces.
%
Our constructive proof of the commutativity rule
also sheds some light on how real commutative implementations might be
built, though the actual construction is not very practical.



\subsection{Actions}

Following earlier work~\cite{herlihy:linearizability}, 
we model a system execution
as a sequence of \emph{actions}, where an action is either an
\emph{invocation} or a \emph{response}.
%
In the context of an operating system, an invocation represents a system
call with arguments (such as \code{getpid()} or \code{open("file", O_RDWR)}) and a
response represents the corresponding result (a PID or a file
descriptor).
%
Invocations and responses are paired. Each invocation is made by a
specific thread, and the corresponding response is returned to the same
thread.
%
An action thus comprises (1) an operation class (e.g., which system
call is being invoked); (2) operation arguments (for invocations) or
a return value (for responses); (3) the relevant thread;
and (4) a tag for uniqueness.
%
We'll write invocations as left half-circles \scpline{\scpi{A}{0}}
(``invoke A'') and responses as right half-circles
\scpline{\scpr{A}{0}} (``respond A''), where the letters match
invocations and their responses.
%
Color and vertical offset differentiate threads:
\scpline{\scpi{A}{1}} and \scpline{\scpi{B}{2}} are invocations on
different threads.

\def\historyexample{%
\scpi{A}{1}
\scpi{B}{3}
\scpi{C}{2}
\scpr{A}{1}
\scpr{C}{2}
\scpr{B}{3}
\scpi{D}{1}
\scpr{D}{1}
\scpi{E}{3}
\scpr{E}{3}
\scpi{F}{2}
\scpi{G}{3}
\scpi{H}{1}
\scpr{F}{2}
\scpr{H}{1}
\scpr{G}{3}}

A system execution is called a \emph{history}. For example:
%
\[H = \text{\scpline{\historyexample}}\]
%
\noindent We'll consider only \emph{well-formed} histories, in which
each thread's actions form a sequence of invocation--response pairs.
$H$ above is well-formed; checking this for the red thread $\scpt{$t$}{1}$, we
see that the thread-restricted subhistory $\HRESTRICT{H}{\scpt{$t$}{1}} =
\text{\scplinerestrictb{1}{\historyexample}}$ formed
by selecting \(\scpt{$t$}{1}\)'s actions from $H$ alternates invocations
and responses as one would want.
%
%\[\HRESTRICT{H}{\scpt{$t$}{1}} = \text{\scplinerestrict{1}{\historyexample}}\]
%\[\HRESTRICT{H}{\scpt{$t$}{x1}} = \text{\scplinerestrictb{1}{\historyexample}}\]
%
%\noindent%
In a well-formed history, each thread has at
most one outstanding invocation at every point.

The \emph{specification} distinguishes whether or not a history is
``correct.''
%
A specification $\SPEC$ is a prefix-closed set of well-formed histories.
%
Its contents depend on the system being modeled; for example, if $\SPEC$
specified a Unix-like OS, then
$[\scpline{\scpi{A}{0}}=\code{getpid()},
\scpline{\scpr{A}{0}}=0] \not\in \SPEC$, since no Unix thread can
have PID 0.
%
Our definitions and proof require that some specification exists, but we
aren't concerned with how it is constructed.


\subsection{Commutativity}
\label{sec:topic:strong-commutativity}

Commutativity should capture the idea that the order of a set
of actions ``doesn't matter.''
%
This happens when \emph{later} actions can't tell which order actually
occurred.
%
The specification helps make this precise:
%
a set of operations commutes in some context when the specification
is indifferent to the execution order of that set.
%
This means that any response valid for one order of the commutative set
is valid for {any} order of the commutative set, and likewise any
response invalid for one order is invalid for any order.
%
But the right definition for commutativity is a little tricky, so we
build it up in two steps.

An action sequence, or region, $H'$ is a \emph{reordering}
of an action sequence $H$ when $\HRESTRICT{H}{t} = \HRESTRICT{H'}{t}$ for
every thread $t$. Thus, regions $H$ and $H'$ contain the same actions, but
may interleave threads differently.
%
If $H =
\text{\scpline{\scpi{A}{1} \scpi{B}{2} \scpr{A}{1} \scpi{C}{1} \scpr{B}{2}
    \scpr{C}{1}}}$, then
%
$\text{\scpline{\scpi{B}{2} \scpr{B}{2} \scpi{A}{1} \scpr{A}{1} \scpi{C}{1}
    \scpr{C}{1}}}$
is a reordering of $H$, but
$\text{\scpline{\scpi{B}{2} \scpi{C}{1} \scpr{B}{2} \scpr{C}{1}
    \scpi{A}{1} \scpr{A}{1}}}$
is not, since it doesn't respect the order of actions in $H$'s red thread.

Consider a history $H = \HX \HCONCAT \HY$ (where $\HCONCAT$ concatenates
action sequences).
%
$\HY$ \emph{\SI-commutes} in $H$ when given any reordering
$\HY'$ of $\HY$, and any action sequence $\HSUFF$,
%
\[ \HX \HCONCAT \HY \HCONCAT \HSUFF \in \SPEC \text{~~~if and only if~~~} \HX \HCONCAT \HY' \HCONCAT \HSUFF \in \SPEC. \]
%
This definition
captures the interface basis and
state dependence we need.  The action sequence $\HX$ puts the system into the state we wish
to consider; switching regions $\HY$ and $\HY'$ requires that the return values from
$\HY$ be valid according to the specification regardless of the actions' order;
and the presence of region $\HSUFF$ in both histories requires that reorderings of
actions in region $\HY$ are indistinguishable by future operations.
%
\XXX[E][There was something about ``This literal transaltion of our
informal definition of commutativity'' here, but it's not ``literal,''
and then I'm not sure we have an informal definition of commutativity?
Maybe the point was to imply this def actually isn't very hard?...]

Unfortunately, \SI commutativity doesn't suffice for the proof
because it is \emph{non-monotonic}.
%
Given an action sequence $\HX \HCONCAT \HY_1 \HCONCAT \HY_2$, it is possible
for $\HY_1 \HCONCAT \HY_2$ to \SI-commute after region $\HX$ even
though $\HY_1$ on its own does not.
%
For example, consider a
\code{get}/\code{set} interface and
%
\(\HY =
  [\scpline{\scpi{A}{1}}=\code{set(1)}, \scpline{\scpr{A}{1}},
   \scpline{\scpi{B}{2}}=\code{set(2)}, \scpline{\scpr{B}{2}},
   \scpline{\scpi{C}{1}}=\code{set(2)}, \scpline{\scpr{C}{1}}]\).
%
\(\HY\) \SI-commutes in any history (every order sets the
underlying value to 2), but its prefix
%
\(\HY_1 =
  [\scpline{\scpi{A}{1}}=\code{set(1)}, \scpline{\scpr{A}{1}},
   \scpline{\scpi{B}{2}}=\code{set(2)}, \scpline{\scpr{B}{2}}]\)
%
does not (some orders set the value to 1 and some to 2).
%
Whether or not $\HY_1$ will
ultimately form part of a commutative region thus depends on \emph{future}
operations!  This is usually incompatible with scalability:
operations in $\HY_1$ must ``plan for the worst'' by remembering their
order in case execution diverges from $\HY_2$.
% \XXX[rtm][is
% ``seems'' implying that appearances are deceiving and it actually
% is compatible with scalability? or do we mean ``there are
% strictly commutative sequences that cannot be made to scale''?
% or ``there
% are strictly commutative sequences for which we do not know of
% scalable implementations''? it would be good to be more clear,
% given that the paragraph's opening claimed
% that strict commutativity does not work. but I feel like we *do*
% know of a scalable implementation of the set(1)/set(2) example,
% which makes me wonder why we need this paragraph at all.]

A monotonic version of commutativity eliminates this problem.
%
An action sequence $\HY$ \emph{\SIM-commutes} in a history $H = \HX \HCONCAT
\HY$ when for any \emph{prefix} $P$ of some reordering of $\HY$ (including
$P = \HY$), $P$ \SI-commutes in $\HX \HCONCAT P$.
%
Like \SI commutativity, \SIM\ commutativity captures 
interface basis and state dependence; unlike \SI commutativity, it
is monotonic and, as we show below, suffices to prove the
commutativity rule.

State dependence means that \SIM\ commutativity captures operations that
commute, and therefore can scale, in some states, but not others.
%
This allows us to greatly expand the situations that commute,
and that therefore can scale.
%
For example, few OS system calls unconditionally commute in
every state and history.
%
(One that does is \code{getpid()}, since its result is constant over a process's
lifetime.)
%
But many system calls \emph{conditionally} commute.
%
Consider Unix's \code{open} system call.
%
Two calls to \code{open("a", O_CREAT|O_EXCL)} often don't
commute: one call will create the file and the other will
fail because the file already exists.
%
However, two calls to \code{open("a", O_CREAT|O_EXCL)} \emph{do}
commute if called from processes with different working
directories.  And even if the processes have the same working
directory, two calls to \code{open("a", O_CREAT|O_EXCL)} will commute
{if the file already exists} (both
calls will return the same error).
%
\SIM\ commutativity allows us to distinguish these cases, even though the
operations are the same in each.  This, in turn, means the commutativity
rule can tell us that scalable implementations exist in the commutative
cases.

\XXX[AC][Talk about how there may not be a single implementation here?
Maybe that's best left to the later discussion, once the formal rule has
been stated.]

\SIM\ commutativity is also interface-based.
%
It evaluates the consequences of execution order using only the
specification.
%
Furthermore, it doesn't say that every reordering has indistinguishable
results on a \emph{given} implementation; it requires instead that
every reordering \emph{is allowed by the specification} to have
indistinguishable results. This is important because any given implementation might
have unnecessary scalability bottlenecks that show through the
interface. The \SIM\ commutativity of an interface can be
considered even when no implementation exists. This in turn makes it
possible to use the commutativity rule early in software development,
during interface design and initial implementation.

\XXX[AC][This differs from what I meant by ``interface-based.''  What I
meant is that there are different interpretations of a system being ``in
the same state'' after any $\HY'$.  It could mean that the internal,
implementation state is the same (down to the bytes, or some other
interpretation), or it could mean the caller can't tell the difference
between two states.  We choose the latter interpretation because it
depends only on the specification, and not on the implementation.]
%
\XXX[E][I see your point, \& tried to capture it.]

\begin{comment}
% Eddie thinks this is not necessary. I also think it IS required
% somewhat by the commutativity rule when you have > 2 threads.
  Finally, \SIM\ commutativity is region-oriented. In contrast with
  typical definitions of commutativity, which consider pairs of
  operations, \SIM\ commutativity applies to any number of threads with
  arbitrary sequences of potentially concurrent operations. This is not
  strictly required by the commutativity rule, but it is a natural
  generalization and it is often convenient to consider likely sequences
  of concurrent operations.
\end{comment}

\XXX[rtm][if my machine has synchronized clocks, do I need ``SRI
commutative''? would local/integral commutative be enough?]
\XXX[E][I believe that's right.]


\subsection{Implementations}

To reason about implementation scalability, we need to model
implementations in enough detail to tell
whether different threads' ``memory accesses'' are conflict-free. (As 
discussed in \S\ref{sec:intro}, conflict freedom is our proxy for
scalability.) We define an implementation as a step
function: given a state and an invocation, it produces a new state and a
response. Special \CONTINUE\ actions enable
concurrent overlapping operations and blocking.

We begin by defining three sets:

\begin{CompactItemize}
\item $S$ is the set of implementation states.
\item $I$ is the set of valid invocations, including \CONTINUE.
\item $R$ is the set of valid responses, including \CONTINUE.
\end{CompactItemize}

An implementation $m$ is a function in $S \times I \mapsto S \times R$.
Given an old state and an invocation, the
implementation produces a new state and a response (where the response
must have the same thread as the invocation).
A \CONTINUE\ response indicates that a real response for that thread is not yet ready,
and allows the implementation to effectively switch to another thread.
\CONTINUE\ invocations give the implementation an opportunity to complete an
outstanding request (or further delay its response); however, the response
must be for the thread matching the \CONTINUE\ invocation.%
\footnote{There are restrictions on how implementation
  arguments are chosen---we assume, for example, that \CONTINUE\ invocations
  are passed only when a thread has an outstanding request.
  Since implementations are functions, they must be
  deterministic. We could model implementations instead as relations,
  allowing non-determinism, though this would complicate later arguments
  somewhat.}
%
% E is not sure this is still important so going to remove it for now
% \XXX[AC][This gives implementations the ability to force sequential
% histories.  As we discussed, this is actually okay because this includes
% what real implementations can really do and so our proof includes
% reality, but that's not something I would expect a reader to put
% together.]
%\XXX[AC][When does an implementation get invoked with \CONTINUE?  Is this
%non-deterministic?  Does it have to be fair?]

An implementation \emph{generates} a history when calls to the
implementation (perhaps including \CONTINUE\ invocations)
could potentially produce the corresponding history.
For example, this sequence
shows an implementation $m$ generating a history
\scpline{\scpi{A}{1} \scpi{B}{2} \scpr{B}{2} \scpr{A}{1}}:

\begin{CompactItemize}
\item $m(s_0, \text{\scpline{\scpi{A}{1}}}) = \langle s_1, \scpt{\CONTINUE}{1} \rangle$
\item $m(s_1, \text{\scpline{\scpi{B}{2}}}) = \langle s_2, \scpt{\CONTINUE}{2} \rangle$
\item $m(s_2, \scpt{\CONTINUE}{1}) = \langle s_3, \scpt{\CONTINUE}{1} \rangle$
\item $m(s_3, \scpt{\CONTINUE}{2}) = \langle s_4, \text{\scpline{\scpr{B}{2}}}\rangle$
\item $m(s_4, \scpt{\CONTINUE}{1}) = \langle s_5, \text{\scpline{\scpr{A}{1}}}\rangle$
\end{CompactItemize}

\noindent The state is threaded from step to step;
invocations appear as arguments and responses as
return values. The generated history consists of the invocations
and responses, in order, with \CONTINUE{}s removed.

An implementation $m$ is \emph{correct} for some specification $\SPEC$
when the responses it generates are always allowed by the specification.
Specifically, assume $H \in \SPEC$ is a valid history and
$r$ is a response where $m$ can generate $H \HCONCAT r$.
We say that $m$ is correct when for any such $H$ and $r$, $H
\HCONCAT r \in \SPEC$. Note that a correct implementation need not be
capable of generating
every possible valid response; it's just that every response it does
generate is valid.

To reason about conflict freedom, we must peek into
implementation states, identify reads and writes, and check for access
conflicts.
%
Let each state $s \in S$ be a tuple $\tup{s.0, \dots, s.m}$, and let
%
$\tupleadj{s}{i}{x}$ indicate component
replacement: $\tupleadj{s}{i}{x} = \tup{s.0,\dots,s.(i-1),x,s.(i+1),\dots,s.m}$.
%
Now consider an implementation step $m(s,a) = \tup{s', r}$.
%
This step \emph{writes} state component $i$
when $s.i \neq s'.i$.
%
It \emph{reads} state component $i$ when $s.i$ may affect the
step's behavior; that is, when for some $y$,
%
\[m(\tupleadj{s}{i}{y},a) \neq \tup{\tupleadj{s'}{i}{y}, r}.\]
%
Two implementation steps have an \emph{access conflict} when they are on
different threads and one writes a state component that the other either
writes or reads.
%
A set of implementation steps is \emph{conflict-free}
when no pair of steps in the set has an access conflict.
%
This notion of access conflicts maps directly onto read and write access
conflicts on real shared-memory machines. Since modern MESI-based
cache-coherent machines
usually provide good scalability on conflict-free access patterns,
we can loosely say that a conflict-free set of implementation steps
``scales.''


\subsection{Rule}

We can now formally state the scalable commutativity rule.
%
Assume a specification $\SPEC$ with a correct reference implementation
$\REFIMP$.
%
Consider a history $H = \HX \HCONCAT \HY$ where $\HY$ \SIM-commutes in $H$,
and where $\REFIMP$ can generate $H$.
%
Then there exists a correct implementation $m$ of $\SPEC$ whose steps in
the $\HY$ region of $H$ are conflict-free.

\XXX[rtm][``the ... $\HY$ region'' seems to imply that $m$ must
generate the same B as $M$.
is that generally true? what if the specification allows more
than one response to an invocation? what if $m$ delays responses
a different amount than $M$? maybe we mean ``there exists an $m$
that generates the same $\HY$ as $M$ and is conflict-free.'']

\XXX[E][In fact the construction does generate the same results as those
in $\HY$, for any ordering of the invocations of $\HY$. And this is true
even if the TRUE implementation, given a different order of invocations
than the original $\HY$ order, WOULD GENERATE DIFFERENT RESULTS. This is
important; it's what I'd like interface-based to mean.]

\XXX[AC][The construction works not just for $\HX \HCONCAT \HY$, but for any
$\HX \HCONCAT \HY'$ where $\HY'$ is a reordering of $\HY$.  Indeed this has to be
the case, since, by its very nature, the construction can't tell the
difference, but we write things as if it has to be $\HY$.  Should we be
clearer that the construction is conflict-free in any $\HY'$?]


\subsection{Proof}

A constructive proof for the commutativity rule can be obtained by
building a scalable
implementation $m$ from the reference implementation $\REFIMP$ and
history $H = \HX \HCONCAT \HY$.
%
The constructed implementation emulates the reference implementation and
is thus correct for any history. Its {performance}
properties, however, are specialized for $H$. For
any history $\HX\HCONCAT P$ where $P$ is a prefix of a
reordering of $\HY$, the constructed implementation's steps in $P$
are conflict-free. This means that, within the \SIM-commutative
region, $m$ scales.


\begin{figure}
\begin{obeylines}
\begin{obeyspaces}
\(\NSIMP(s, a) \equiv\)
~~~If \(\HEAD(s.h) = a\):
~~~    \(r \gets \CONTINUE\)
~~~else if \(a = \CONTINUE\) and \(\HEAD(s.h)\) is a response
~~~              and \(\THREAD(\HEAD(s.h)) = \THREAD(a)\):
~~~    \(r \gets \HEAD(s.h)\)   \COMMENT{replay $s.h$}
~~~else if \(s.h \neq \EMULATE\):   \COMMENT{$H$ complete or input diverged}
~~~    \(H' \gets \text{an invocation sequence consistent with \(s.h\)}\)
~~~    For each invocation \(x\) in \(H'\):
~~~        \(\tup{s.\V{refstate}, \_} \gets \REFIMP(s.\V{refstate}, x)\)
~~~    \(s.h \gets \EMULATE\)   \COMMENT{switch to emulation mode}
~~~If \(s.h = \EMULATE\):
~~~    \(\tup{s.\V{refstate}, r} \gets \REFIMP(s.\V{refstate}, a)\)
~~~else:                 \COMMENT{replay mode}
~~~    \(s.h \gets \text{tail}(s.h)\)
~~~return \(\tup{s, r}\)
\end{obeyspaces}
\end{obeylines}
\rule{\columnwidth}{0.5pt}
\vspace{-\baselineskip}
\caption{Constructed \emph{non-scalable} implementation $\NSIMP$ for history $H$ and
  reference implementation $M$.}
\label{fig:nonscalablesim}
\end{figure}

% Keep an index in $H$ instead of a mutable $s.h$?  Explicitly represent
% emulation mode in that index?  Messy with \END\ and per-thread $s.h$ in
% $m$.  Explicitly represent mode of $m$?  Could have a single commute
% flag in $m$, set when you reach the end of $A$.

To understand the construction, it helps to first imagine constructing a
\emph{non-scalable} implementation $\NSIMP$ from the reference $M$.
%
This non-scalable implementation begins in \emph{replay mode}.
%
While the input invocations match $H$, $\NSIMP$ responds exactly according to
$H$, without invoking the reference.
%
When the input invocations diverge from $H$, however, $\NSIMP$ no
longer knows how to respond, so it enters \emph{emulation mode}.
%
This requires feeding $M$ all previously received invocations to
prepare its state.

A state $s$ for $\NSIMP$ contains two components.
%
First, $s.h$ holds either the portion of $H$ that remains to be replayed
or \EMULATE, which denotes emulation mode.
It is initialized to $H$.
%
Second, $s.\V{refstate}$ is the state of the reference
implementation, and is initialized accordingly.
%
Figure~\ref{fig:nonscalablesim} shows how the simulated implementation
works.
%
We make several simplifying assumptions, including that $\NSIMP$
receives \CONTINUE\ invocations in a restricted way; these assumptions aren't
critical for the argument.
%
One line requires expansion, namely the choice of $H'$ ``consistent
with \(s.h\)'' when the input sequence diverges.
%
This step calculates the prefix of $H$ up to, but not including, $s.h$;
excludes responses; and adds \CONTINUE\ invocations as appropriate.

This implementation is correct---its responses for any history
always match those from the reference implementation.
%
But it doesn't scale.
%
In replay mode, any two steps of $\NSIMP$ conflict on accessing
$s.h$. These accesses track which invocations
have occurred; without them it would be impossible to later
initialize the state of $M$.
%
And this is where commutativity comes in.
%
The action order in a \SIM-commutative region doesn't matter by
definition.
%
Since the specification doesn't distinguish among orders, it
is safe to initialize the reference implementation with the commutative
actions \emph{in a different order than they were received}.
%
All future responses will still be valid according to the specification.

\begin{figure}
\begin{obeylines}
\begin{obeyspaces}
\(m(s, a) \equiv\)
~~~\(t \gets \THREAD(a)\)
~~~If \(\HEAD(s.h[t]) = \COMMUTE\):  \COMMENT{enter conflict-free mode}
~~~    \(s.\V{commute}[t] \gets \TRUE\);~ \(s.h[t] \gets \TAIL(s.h[t])\)
~~~If \(\HEAD(s.h[t]) = a\):
~~~    \(r \gets \CONTINUE\)
~~~else if \(a = \CONTINUE\) and \(\HEAD(s.h[t])\) is a response
~~~              and \(\THREAD(\HEAD(s.h[t])) = t\):
~~~    \(r \gets \HEAD(s.h[t])\)   \COMMENT{replay $s.h$}
~~~else if \(s.h[t] \neq \EMULATE\):   \COMMENT{$H$ complete/input diverged}
~~~    \(H' \gets \text{an invocation sequence consistent with \(s.h[\ast]\)}\)
~~~    For each invocation \(x\) in \(H'\):
~~~        \(\tup{s.\V{refstate}, \_} \gets \REFIMP(s.\V{refstate}, x)\)
~~~    \(s.h[u] \gets \EMULATE\) for each thread \(u\)
~~~If \(s.h[t] = \EMULATE\):
~~~    \(\tup{s.\V{refstate}, r} \gets \REFIMP(s.\V{refstate}, a)\)
~~~else if \(s.\V{commute}[t]\):   \COMMENT{conflict-free mode}
~~~    \(s.h[t] \gets \text{tail}(s.h[t])\)
~~~else:   \COMMENT{replay mode}
~~~    \(s.h[u] \gets \text{tail}(s.h[u])\) for each thread \(u\)
~~~return \(\tup{s, r}\)
\end{obeyspaces}
\end{obeylines}
\rule{\columnwidth}{0.5pt}
\vspace{-\baselineskip}
\caption{Constructed scalable implementation $m$ for history $H$ and
  reference implementation $M$.}
\label{fig:scalablesim}
\end{figure}

Figure~\ref{fig:scalablesim} shows the construction of $m$, a
version of $M$ that scales over $\HY$ in $H = \HX\HCONCAT \HY$.
%
$m$ is similar to $\NSIMP$, but extends it with a \emph{conflict-free
  mode} used to execute actions in $\HY$.
%
Its state is as follows:

\begin{CompactItemize}
\item $s.h[t]$---a per-thread history. Initialized to
  $\HX\HCONCAT\COMMUTE\HCONCAT(\HRESTRICT{\HY}{t})$, where the special
  \COMMUTE\ action indicates the commutative region has begun.
\item $s.\V{commute}[t]$---a per-thread flag indicating whether the
  commutative region has been reached. Initialized to \FALSE.
\item $s.\V{refstate}$---the reference implementation's state.
\end{CompactItemize}

\noindent%
Each step of $m$ in the commutative region \emph{accesses
only state components specific to the invoking thread}.
%
This means that any two steps in the commutative region are
conflict-free, and the commutativity rule is proved.
%
The construction uses \SIM\ commutativity when initializing the
reference implementation's state via $H'$.
%
If the observed invocations diverge before the commutative region, then
just as in $\NSIMP$, $H'$ will exactly equal the observed invocations.
%
If the observed invocations diverge in or after the commutative region,
however, there's not enough information to recover the order of
invocations. (The $s.h[t]$ components track which invocations have
happened per thread, but not the order of those invocations between
threads.) Therefore, $H'$ might reorder the invocations in $\HY$.
%
\SIM\ commutativity guarantees that replaying $H'$ will nevertheless produce
results indistinguishable from those of the actual invocation order,
even if the execution
diverges \emph{within} the commutative region.\footnote{%
  We effectively have assumed that $M$, the reference
  implementation, produces the same results for any reordering of the
  commutative region. This is stricter than \SIM\ commutativity, which
  places requirements on the {specification}, not the
  implementation. We also assumed that $M$ is indifferent to the
  placement of \CONTINUE\ invocations in
  the input history.  Neither of these restrictions is fundamental, however.  If
  during replay $M$ produces responses that are inconsistent with the
  desired results, $m$ could throw away $M$'s state, produce a new $H'$
  with different \CONTINUE\ invocations and/or commutative region ordering,
  and try again.
  This procedure must eventually succeed.}

\begin{comment}
\begin{figure*}
% A serial history
(a) \scpline{\scpi{A}{1}
\scpr{A}{1}
\scpi{B}{1}
\scpr{B}{1}
\scpi{C}{1}
\scpr{C}{1}
\scpi{D}{1}
\scpr{D}{1}
\scpi{E}{1}
\scpr{E}{1}
\scpi{F}{1}
\scpr{F}{1}
\scpi{G}{1}
\scpr{G}{1}} \dots

\vskip5pt
% A three-threaded history
\def\threethreadedhistory{%
\scpi{A}{1}
\scpi{B}{3}
\scpi{C}{2}
\scpr{A}{1}
\scpr{C}{2}
\scpr{B}{3}
\scpi{D}{1}
\scpr{D}{1}
\scpi{E}{3}
\scpr{E}{3}
\scpi{F}{2}
\scpi{G}{3}
\scpi{H}{1}
\scpr{F}{2}
\scpr{H}{1}
\scpr{G}{3}
\scpi{I}{2}
\scpr{I}{2}
\scpi{J}{2}
\scpi{K}{1}
\scpr{J}{2}
\scpi{L}{3}
\scpr{K}{1}
\scpi{M}{1}
\scpr{L}{3}
\scpr{M}{1}}
(b) \scpline{\threethreadedhistory} \dots

\vskip5pt
\begin{tabular}{@{}l@{~}l@{}}
(c) & \scplinerestrict{1}{\threethreadedhistory}\\
& \scplinerestrict{2}{\threethreadedhistory}\\
& \scplinerestrict{3}{\threethreadedhistory}
\end{tabular}

\vskip5pt
(d) \scpline{\scpix{A}{3}
\scpix{B}{1}
\scprx{A}{3}
\scpix{C}{3}
\scprx{C}{3}
\scpix{D}{2}
\scpix{E}{3}
\scprx{B}{1}
\scprx{E}{3}
\scprx{D}{2}
\scpi{F}{3}
\scpi{G}{1}
\scpi{H}{2}
\scpr{G}{1}
\scpr{F}{3}
\scpr{H}{2}
\scpi{I}{1}
\scpi{J}{2}
\scpr{J}{2}
\scpix{K}{3}
\scprx{I}{1}
\scprx{K}{3}
\scpix{L}{3}
\scpix{M}{2}
\scprx{L}{3}
\scpix{N}{3}
\scprx{M}{2}
} \dots

\def\commutativeregiona{%
\scpi{F}{3}
\scpi{G}{1}
\scpi{H}{2}
\scpr{G}{1}
\scpr{F}{3}
\scpr{H}{2}
\scpi{I}{1}
\scpi{J}{2}
\scpr{J}{2}}

\def\commutativeregionb{%
\scpi{H}{2}
\scpi{F}{3}
\scpr{H}{2}
\scpi{G}{1}
\scpi{J}{2}
\scpr{F}{3}
\scpr{G}{1}
\scpr{J}{2}
\scpi{I}{1}}

\def\commutativeregionc{%
\scpi{G}{1}
\scpi{H}{2}
\scpi{F}{3}
\scpr{H}{2}
\scpr{G}{1}
\scpi{I}{1}}


\vskip5pt
\begin{tabular}{@{}l@{~}l@{\qquad\qquad}l@{\qquad\qquad}l@{}}
(e) & \scpline{\commutativeregiona}
   & \scpline{\commutativeregionb}
   & \scpline{\commutativeregionc} \\
\noalign{\vskip3pt}
   & \scplinerestrict{1}{\commutativeregiona}
   & \scplinerestrict{1}{\commutativeregionb}
   & \scplinerestrict{1}{\commutativeregionc} \\
   & \scplinerestrict{2}{\commutativeregiona}
   & \scplinerestrict{2}{\commutativeregionb}
   & \scplinerestrict{2}{\commutativeregionc~\scpicut{J}{2}~\scprcut{J}{2}} \\
   & \scplinerestrict{3}{\commutativeregiona}
   & \scplinerestrict{3}{\commutativeregionb}
   & \scplinerestrict{3}{\commutativeregionc~\scprcut{F}{3}} \\
\end{tabular}

\caption{\textbf{(a)} A serial, single-threaded history. Each invocation (filled
  circle) is immediately followed by its response (open circle).
%
  \textbf{(b)} A three-thread history. Colors indicate different threads.
%
  \textbf{(c)} The thread-restricted subhistories of (b). Each is
  serial.
%
  \textbf{(d)} A three-thread history with a commutative region. The
  commutative region is highlighted.
%
  \textbf{(e)} Three interleavings of the commutative region.
%
  \protect\XXX[E][This figure is currently unreferenced, leaving it in case
  others think it useful.]~
}
\end{figure*}
\end{comment}

\subsection{Discussion}

The commutativity rule and proof construction push state and
history dependence to an extreme: the proof construction is specialized
for a single commutative region.
%
Repeated application of the construction can build an
implementation that scales over multiple commutative regions in a
history, or for the union of many histories.
%
(This is because, once the constructed machine leaves the specialized
region, it passes invocations directly to the reference and has
the same conflict-freedom properties as the reference.)
%
Nevertheless, the proof construction is impractical, and real
implementations usually achieve scalability using different techniques.

We believe it is easier to create practical scalable implementations for
operations that commute in more situations.
%
The arguments and system states for which a set of operations
commutes often collapse into
fairly well-defined classes (e.g., file creation might
commute whenever the containing directories are different).
%
In practice, implementations scale for whole classes of states and
arguments, not just for specific histories.

It is also often the case that a set of operations
commutes in more than one class of situation,
% and while there exist scalable implementations for each class, 
but no
single implementation scales for all classes.
%
Consider, for example, an interface with two calls:
\code{put(}$x$\code{)} records a sample with value $x$, and
\code{max()} returns the maximum sample recorded so far (or \code{0}).
Suppose
%
\[H = [\scpline{\scpi{A}{1}}=\code{put(1)}, \scpline{\scpr{A}{1}},
   \scpline{\scpi{B}{2}}=\code{put(1)}, \scpline{\scpr{B}{2}},
   \scpline{\scpi{C}{3}}=\code{max()}, \scpline{\scpr{C}{3}}=1].
\]
%
\XXX[E][We put other similar histories inline and this uses lots of space]
%
An implementation could store per-thread maxima reconciled by
\code{max} and be conflict-free for $\scpline{\scpir{A}{1}
  \scpir{B}{2}}$ in $H$.  Alternatively, it could use a
global maximum that \code{put} checked before writing.  This
is conflict-free for $\scpline{\scpir{B}{2} \scpir{C}{3}}$ in $H$.
%Symmetric cases hold in the history $\scpline{\scpir{B}{2} \scpir{A}{1}
%  \scpir{C}{3}}$. 
%% E: I am happy to ignore this nit; not sure it's relevant.
%
But no correct implementation can be conflict-free across all of $H$.
%
In the
end, a system designer must decide which situations involving
commutative operations are most important, and find practical
implementation strategies that scale in those situations.
%
In \S\ref{sec:model} we show that many operations in POSIX have
implementations that scale quite broadly, with few cases of incompatible
scalability classes.

%It may also be possible to implement non-commutative operations
%scalably. _------THAT IS REPET WITH THE BELOW
%
The commutativity rule shows that \SIM-commutative regions
have conflict-free implementations.
%
It does not show the converse, however:
commutativity \emph{suffices} for conflict-free accesses,
but it may not be \emph{necessary}.
%
Some non-commutative interfaces may have scalable
implementations---for instance, on machines that
offer scalable access to strictly increasing sources of time, or when
the core interconnect allows certain communication patterns to scale.
%
Furthermore, some conflict-free access patterns don't scale on real
machines; if an application overwhelms the memory bus with memory accesses,
scalability will suffer regardless of whether those accesses have
conflicts.
%
We hope to investigate these problems in future, but as we show below,
the rule is already a good guideline for achieving practical
scalability.


% The rest of this paper gives examples of how the rule guided
% us toward missed opportunities for scalability, and inspired new
% scalable implementation strategies.

\begin{comment}
\section {Old version of The commutativity rule}

The basic (but imprecise) commutativity rule says that if two operations
commute, then a scalable implementation is possible.  The intuition
behind the rule is that if two operations yield the same results
regardless of order, there can't be any need for one to communicate
to the other.
% If operation $A$ executes first, it must be possible for operation $B$ to
% execute without learning anything from $A$'s execution, because $B$
% could have gone first.
Per the argument of conflicted-access parallelism,
if they don't need to communicate, a scalable implementation must be
possible.  The basic rule seems obviously true for operations that
commute unconditionally; for example, two \code{getpid} calls.

Operating systems, however, have many calls that do not commute
unconditionally, but only in certain states or with certain arguments.
%
For example, consider Unix's \code{open} system call with the
\code{O_CREAT|O_EXCL} flags, which creates a file iff the file doesn't
already exist.
%
When two processes concurrently use \code{open(O_CREAT|O_EXCL)} to
create files with the same name, at most one of them will succeed.
%
Not only does this mean \code{open} doesn't commute in general, it also
implies that a correct implementation of \code{open} \emph{cannot be
scalable} according to conflicted-access parallelism (some memory location
representing the file must be shared).

\code{open} does commute, however, in many cases.
%
For example, two
\code{open}s without \code{O_EXCL} usually commute,
because in either order they will usually produce the same result.
Two calls to \code{open(name, O_CREAT|O_EXCL)} will commute
if the \code{name}s are different.  Even if the names are the
same, two calls to \code{open(name, O_CREAT|O_EXCL)} will commute if
the name is a relative path and the two threads have different
working directories. And even if the threads have the same working
directory, two calls to \code{open(name, O_CREAT|O_EXCL)} with the same
\code{name} will commute \emph{if the file already exists} (both system
calls will return the same error).

This section provides a precise commutativity rule that allows a
designer to reason about these different cases of the \code{open} call,
so that a designer can exploit the commutative cases for a scalable
implementation.  In particular, what we show is that given any state
where an operation (with specific arguments)
\emph{does} commute with one or more others, an
implementation exists that is scalable \emph{in that state, over those
  operations}. 

\subsection{Interfaces, implementations and scalability}
\label{pf}

To make the rule precise, we need a way to talk about interface
specifications, implementations, and scalability.  This subsection
introduces the necessary definitions.

Our notions for
interface specifications and for implementations draw from the action
calculus \XXX[E] used by Herlihy and Wing to present
linearizability~\cite{herlihy:linearizability}.
%
We model interfaces and implementations via \emph{histories}, which are
sequences of operation invocations and responses.
%
An invocation is a call to an interface function, such as \code{getpid},
including any arguments.
%
A response is the return value.
%
Each invocation and response belongs to a \emph{thread}, which is
written with a subscript: $\ATT{\code{getpid}}{t}$ represents an
invocation of \code{getpid} by thread $t$.

A history combines invocations and responses from many threads.
%
We can restrict a history $\HIST{H}$ to a single thread $t$, creating
the sub-history $\HRESTRICT{\HIST{H}}{t}$, by extracting the invocations
and responses for that thread.
%
In a well-formed history, each sub-history must be \emph{serial},
meaning that it looks like the trace of a single-threaded program: an
alternating series of invocations and the corresponding responses, with
at most one outstanding invocation at the end.
%
$[\ATT{\code{getpid}}{t}, \ATT{192}{t},
\ATT{\code{getpid}}{t}, \ATT{192}{t}]$ is a well-formed serial history,
but $[\ATT{192}{t}, \ATT{\code{getpid}}{t}, \ATT{\code{getpid}}{t},
\ATT{192}{t}]$ is not.

A \emph{specification} $\SPEC$ is a set of \emph{legal} histories that
defines the allowed behavior for an interface.
%
For example, assume $\SPEC$ defines the behavior of \code{getpid} on a
system where each process has exactly one thread.
%
Then $[\ATT{\code{getpid}}{t}, \ATT{1}{t}] \in \SPEC$ and
$[\ATT{\code{getpid}}{t}, \ATT{2}{t}] \in \SPEC$ (a thread can have any
PID).
%
But a PID can't be zero, so $[\ATT{\code{getpid}}{t}, \ATT{0}{t}]
\not\in \SPEC$;
%(a PID
a thread's PID can't change, so $[\ATT{\code{getpid}}{t}, \ATT{1}{t},
\ATT{\code{getpid}}{t}, \ATT{2}{t}] \not\in \SPEC$;
%
and distinct threads must have distinct PIDs, so
$[\ATT{\code{getpid}}{t}, \ATT{\code{getpid}}{u},
\ATT{1}{t}, \ATT{1}{u}] \not\in \SPEC$.
%
Every specification must contain only well-formed histories, and every
specification must be prefix-closed.
% if $\HIST{H}$ is a prefix of
% $\HIST{H'}$ and $\HIST{H'} \in \SPEC$, then $\HIST{H} \in \SPEC$.

An \emph{implementation} can be thought of as
a procedure that generates responses for
outstanding invocations in a history.
%
We write an implementation as a step relation $\STEP$.
%
If $H \STEP H\HCONCAT r$, then $H$ is a history with one or more
outstanding invocations, and $r$ is a response to one of the outstanding
invocations.
%
Different implementations might generate different responses, and
responses need not be deterministic.
%---need not be unique (a single implementation might allow both
%$P \STEP P\HCONCAT r$ and $P \STEP P\HCONCAT r'$ with $r \neq r'$).
%
A \emph{correct} implementation will ensure that every generated
response is legal according to its specification $\SPEC$.
%
Specifically, an implementation is correct if $P\HCONCAT r \in \SPEC$
whenever $P \in \SPEC$ and $P \STEP P\HCONCAT r$.

Note that an implementation has no say over the order in which
invocations arrive.
%
This depends only on the specification.
%
In most situations, such as operating system kernels, a specification
must allow any invocation at any time (though many possible invocations
will generate error responses).

This formalization captures any system state that is exposed at the
interface level.
%
For example, a specification for a shared variable with \code{get} and
\code{put} operations would effectively require an implementation to
keep shared state, since the specification would require each \code{get}
response to contain the value most recently \code{put}.

We define scalability in terms of disjoint-access parallelism
using the following abstract machine.  Each core effectively has an
infinite local cache that is kept consistent with other local caches
using a coherence protocol. \emph{Scalable} means that operations on
different threads do not communicate using shared memory: a memory
location written by the implementation of thread $t$ is not read
by any other thread.  That is, for every pair of cores, the two
cores access either completely disjoint memory locations, or if they do
share any memory, both cores only read the shared memory. As a result, a
set of operations that scales will not experience coherence misses;
rather, it will experience only cache misses it would have experienced
when executing in isolation or on a single core.  This abstract machine
is simpler than any actual machine,\footnote{Roughly speaking, a real
  machine fits the abstract model as long as a workload's working set
  fits in cache, the interconnect and DRAM interface are not
  bottlenecks, and the machine uses a modern scalable directory-based
  coherence protocol.} but \S\ref{sec:eval} confirms that when
operations have disjoint-access parallelism, they typically scale perfectly on
real hardware.

\subsection{Interleavings and commutativity}

A \emph{region} is a contiguous subsequence of actions in some
well-formed history.
%
A region $X$ is an \emph{interleaving} of a region $Y$
if their per-thread sub-histories are identical.
%
This allows $X$ to reorder the actions of $Y$ across threads, but not
within them.
%
Using our notation, $X$ is an interleaving of $Y$ if $\HRESTRICT{X}{t} =
\HRESTRICT{Y}{t}$ for all $t$.

Informally, operations commute if the specification can't
distinguish their order:
%
changing the interleaving of some commutative region shouldn't
change the legality of a history.
%
But this informal statement misses an important property.
%
We want to consider operations commutative based on their behavior in
some state and given some \emph{past} history.
%
But interleavings also constrain \emph{future} operations, since every
interleaving of a region contains \emph{all} of that region's
operations.
%
We thus require that for commutative $X$, the specification can't
distinguish among orderings of arbitrary \emph{prefixes} of
interleavings of $X$.
%
This captures the idea that an implementation should scale throughout a
commutative region, regardless of whether future invocations from that
region actually arrive.

Formally, consider a region $X$ of a history $H$,
%
with $H = A \HCONCAT X \HCONCAT B \in \SPEC$.
%
Let $Z$ be a prefix of some interleaving of $X$. (That is,
$\HRESTRICT{Z}{t}$ is a prefix of $\HRESTRICT{X}{t}$ for all $t$.)
%
Further let $Z'$ be an interleaving of $Z$.
%
Finally, let $B'$ be an arbitrary sequence of actions.
%
Then $X$ \emph{commutes} in $H$ when the following conditions hold for
all $Z$, $Z'$, and $B'$:
%
(1) $A \HCONCAT Z \in \SPEC$; and
%
(2) $A \HCONCAT Z \HCONCAT B' \in \SPEC$ if and only if $A \HCONCAT Z' \HCONCAT B'
\in \SPEC$.

\XXX[Xi Wang][Want to see an example here.]

\subsection{The rule}
\label{rulerule}

The commutativity rule is precisely stated as follows.
%
Assume there exists a correct implementation of $\SPEC$.
%
Then for any $\HIST{H} \in \SPEC$ and region $X$ that commutes in
$\HIST{H}$,
%
there exists a correct implementation $R$ of $\SPEC$ that is
\emph{scalable} over the operations in $H$ wholly contained in $X$.
%

\XXX[Xi Wang][Want to see an example here too.]

The intuition behind the precise commutativity rule is simple.
%
Since $X$ commutes in $\HIST{H}$, the implementations of the
operations in $X$ don't need to communicate to determine their order,
since their responses are the same regardless of order and later
invocations cannot reveal their order.

We formalize this intuition by constructing an
implementation $M$
whose threads don't communicate in $X$.
%
This implementation is tailored for the history $\HIST{H}$ and
commutative region $X$. Though it is correct for every history, it is
guaranteed to {scale} only for $X$ in $\HIST{H}$.
%
Conceptually, all threads of $M$ execute operations synchronously and in
lockstep as long as the received invocations follow $\HIST{H}$.  If
execution diverges from $\HIST{H}$, $M$ simply falls back to executing
the reference implementation $R$.  However, if execution reaches $X$
without diverging from $\HIST{H}$, then $M$ enters concurrent mode, in
which each thread independently replays just its own responses from $X$.
Here, $M$ may generate responses in a different order than $\HIST{H}$,
but thanks to commutativity any such order is legal.  In concurrent
mode, no synchronization is necessary, so the threads of $M$ do not
communicate.

% $M$ compares received invocations against an internal copy of $H$.
% %
% If its received invocations match $H$,
% then $M$ plays back the responses from its copy of $H$.
% %
% If they don't match, then $M$ is seeing a sequence of operations other
% than the one it's tailored for, and $M$ switches over to simulating a
% correct reference implementation, $R$, which we assume exists.
% %
% This requires replaying the previous invocations to prepare $R$'s
% internal state.

% Since $M$ is simply playing back responses, you might
% expect it to scale over all of $H$.
% %
% It doesn't, though, since tracking received invocations is not scalable.
% %
% Global synchronization is required to ensure that all
% threads' received invocations match those in $H$.
% %
% Except, that is, in the commutative region $X$, where
% by definition, \emph{any} replay order must create an equivalent
% result. $M$ can switch to unsynchronized, scalable tracking upon
% reaching $X$.

%% Without further assumptions on the nature of the interface being
%% represented, a specifically tailored implementation is the best we can
%% do.

%% By assumption a correct reference implementation $R$ exists.
%
Figure~\ref{fig:proof} gives the detailed construction of $M$ from a
reference implementation $R$ and a legal history $\HIST{H} = A \HCONCAT X$
where $X$ commutes in $\HIST{H}$.
%
% Then an implementation that scales over $X$ when given a history
% beginning with $A\HCONCAT X$, but returns the same result as the reference
% implementation for \emph{any} history, can be constructed by tracking
% received invocations against $A\HCONCAT X$.
% %
% Within $A$, invocation tracking must be non-scalable; but within $X$, it
% can be scalable, thanks to commutativity.
% %
% When received invocations either complete $A\HCONCAT X$ or diverge from it,
% the constructed implementation replays the received invocations to the
% reference implementation.
%
$M$ augments $R$'s state:
for every thread $t$, we also
store a history $H_t = A \HCONCAT (\HRESTRICT{X}{t})$
and a history program counter $\textit{pc}_t$.
Each $\textit{pc}_t$
initially points at the first invocation in $H_t$. The value
$\textit{pc}_t = \FAIL$ indicates the thread should fall back to the
reference implementation.

\begin{figure}
\rule{\columnwidth}{0.5pt}
  \begin{CompactEnumerate}
\item If invocation $i$ is received on thread $t$, and $\textit{pc}_t$
  points to an invocation $I$ that also belongs to thread $t$, then:
 \begin{CompactEnumerate}
\item If $i = I$ and $\textit{pc}_t$ is in $A$, then the thread
  non-scalably increments all $\textit{pc}$ values, including its own.
\item If $i = I$ and $\textit{pc}_t$ is in $\HRESTRICT{X}{t}$, then the
  thread scalably increments \emph{only its own} $\textit{pc}_t$.
\item If $i \neq I$, then the thread non-scalably falls back to the
  reference implementation $R$. This involves pausing all other threads,
  taking a snapshot of all $\textit{pc}_t$ values, replaying the
  invocations indicated by the \textit{pc}s on $R$ (while throwing away
  the responses from $R$, which must equal those already generated by $M$),
  and finally setting all $\textit{pc}_t$ values to \FAIL. Care is
  required to ensure that only one thread completes the fallback.
%
\end{CompactEnumerate}
\item If $\textit{pc}_t$ points to a response that belongs to thread $t$, then the
  implementation generates that response and increments $\textit{pc}_t$.
  If this response is in $A$, the implementation also
  non-scalably increments the other threads' \textit{pc} values.
\item If $\textit{pc}_t$ points beyond $H_t$, then $t$ falls back to $R$
  as described above.
\item If $\textit{pc}_t$ points at an invocation or response that does not belong to
  thread $t$, then $t$ waits. Eventually another thread will either
  advance $\textit{pc}_t$ or set $\textit{pc}_t$ to \FAIL.
\item If $\textit{pc}_t = \FAIL$, then $t$ simulates $R$.
\end{CompactEnumerate}
\vspace{-\baselineskip}
\vspace{1ex}
\rule{\columnwidth}{0.5pt}
\vspace{-\baselineskip}
\caption{Simulation procedure for each thread $t$ of $M$.}
\label{fig:proof}
\end{figure}

Before $M$ reaches the commutative region, it
%
exactly replays $H$'s responses in order using lockstep \textit{pc} updates.
%
% After reaching the commutative region, $M$ may generate responses in a
% different order than $H$, but thanks to commutativity any such order is
% legal.
%
When the input history diverges from $H$, the implementation
replays the received invocations to the reference implementation.
%
For invocations in the commutative region $X$, the replay order may
differ from the actual order in which the original invocations were
received, because a scalable implementation cannot record a global
reception order,
but since $X$ is commutative, any replay order is allowed since later
invocations can't distinguish among them.
%
(We assume without loss of generality that $R$ can be made
deterministic; that is, given any interleaving $X'$ of $X$ and any
prefix $P$ of $A \HCONCAT X'$, we assume that $R$'s responses to $P$'s invocations
during replay are the same as the actual responses in $P$. This could if
necessary be simulated by passing the expected responses to $R$.)
%
Within the commutative region $X$, each thread of $M$ reads and
writes only local state ($\textit{pc}_t$ and $H_t$), meaning the threads
do not communicate using shared memory.
%
Thus, $M$ scales over $X$ and is correct for any sequence of
invocations. \qed
\XXX[AC][Are we allowed to \qed without an opening \textbf{Proof}?]
\XXX[EK][NO WE WILL BE KILLED]

\subsection{Discussion}

The commutativity rule allows us to talk about operations that commute
with specific arguments and in specific system states.  This ability is
powerful, because even when an interface commutes in limited cases, the
rule suggests that we may be able to find
a scalable implementation for those cases.

The rule identifies two valuable degrees of freedom that a developer can
exploit.  The developer may be able to change the
specification to make it commute in more situations.  The commutativity
rule allows a designer to evaluate the potential scalability of
interfaces without having to first implement them.

Another degree of freedom is in implementations. The proof deals with a
limited window of operations, and says that scalability is possible
within that window. That suggests that it may be possible for the
implementation to push the non-scalability to outside the window. This
turns out to be useful when deferring non-scalable work allows efficient
batching. \XXX[EK][don't care about this point.] \XXX[FK][Why not?  We
  use it often.]

There will usually be an open-ended set of arguments and system states
for which some operations commute. Luckily these sets usually collapse
into fairly well-defined classes (e.g., concurrent file creation might
commute whenever the containing directories are different), so that
implementations need to be tailored only for the class, and not for
the specific arguments.

On the other hand, it is often the case that a set of operations
commutes in more than one class of situation. While there may be a
scalable implementation for each class, there may not be a convenient
combined implementation that scales for all of the classes.  In the end,
a system designer must decide which situations involving commutative
operations are most important, and find practical implementation
strategies that scale in those situations.  

The rest of this paper gives examples of how the rule guided
us toward missed opportunities for scalability, and inspired new
scalable implementation strategies.

\end{comment}
