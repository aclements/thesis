\section{Conclusion}
\label{sec:concl}

\XXX[STATUS]{Unchanged from SOSP.}

\XXX![AC]{Needs expansion.}

The scalable commutativity rule provides a new approach for software
developers to understand and exploit multicore scalability starting at
the software interface.
%
We defined \SIM commutativity, which allows developers to apply the
rule to complex, stateful interfaces.
%
We further introduced \tool to help programmers analyze interface
commutativity and test that an implementation scales in commutative
situations.
%
Finally, using \sys, we showed that it is practical to achieve a
broadly scalable implementation of POSIX by applying the rule, and
that commutativity is essential to achieving scalability and
performance on real hardware.
%
We hope that programmers will find the commutativity rule helpful to
produce software that is scalable by design.

\tool, \sys, and a browser for the data in this dissertation are
available at
\url{http://pdos.csail.mit.edu/commuter}.
