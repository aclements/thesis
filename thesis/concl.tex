\section{Conclusion}
\label{sec:concl}

\XXX[STATUS]{v2: Tweaked.  v1: Unchanged from SOSP.}

This dissertation introduced a new approach for software developers to
understand and exploit multicore scalability during software interface
design, implementation, and testing.
%
We defined, formalized, and proved the scalable commutativity rule,
the key observation that underlies this new approach.
%
We defined \SIM commutativity, which allows developers to apply the
rule to complex, stateful interfaces.
%
We further introduced \tool to help programmers analyze interface
commutativity and test that an implementation scales in commutative
situations.
%
Finally, using \sys, we showed that it is practical to achieve a
broadly scalable implementation of POSIX by applying the rule, and
that commutativity is essential to achieving scalability and
performance on real hardware.

% We hope that programmers will find the scalable commutativity rule helpful to
% produce software that is scalable by design.


% Looking forward, as scalability becomes increasingly important at all
% levels of the software stack, we hope that the scalable commutativity
% rule will help shape the way developers design and implement scalable
% software.


% Scalability is becoming increasingly important at all levels of the
% software stack and the software interfaces we define today will have
% to cope with the dramatically more parallel machines of tomorrow.
% %
% We hope that the scalable commutativity rule will shape the way
% developers design and implement scalable software.


We are in the midst of a sea change in software performance, as
everything from top-tier servers to embedded devices turns to
increasing parallelism to maintain a competitive performance edge.
%
As scalability becomes increasingly important at all levels of the
software stack, we hope that the scalable commutativity rule will help
shape the way developers meet this challenge.
