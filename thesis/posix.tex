\section{Designing commutative interfaces}
\label{sec:posix}

\XXX[STATUS]{Unchanged from SOSP.  No changes planned.}

\noindent
The rule facilitates scalability reasoning at the interface and
specification level, and \SIM\ commutativity lets us apply the rule to
complex interfaces.
%\Cref{sec:topic:strong-commutativity} gave a glimpse of this using
%\code{open}.
This section demonstrates the interface-level reasoning enabled by the
rule. Using POSIX as a case study, we explore changes that make operations
commute in more situations, enabling more scalable implementations.
%
Already, many POSIX operations commute with many other
operations, a fact we will quantify in the next section; this
section focuses on
problematic cases to give a sense of the subtler issues of commutative
interface design.
%
% While the details here are specific to POSIX, the general ideas are
% applicable to any interface.

\paragraph{Decompose compound operations.}

Many POSIX APIs combine several operations into one, limiting the
combined operation's commutativity.
For example,
\code{fork} both creates a new process and snapshots the current process's
entire memory state, file descriptor state, signal mask, and several
other properties.  As a result, \code{fork} fails to commute
with most other operations in the same process, such as memory writes,
address space operations, and many file descriptor operations.
However, applications often follow \code{fork} with
\code{exec}, which undoes most of \code{fork}'s sub-operations.
With only \code{fork} and \code{exec}, applications are forced to accept
these unnecessary sub-operations that limit commutativity.

POSIX has a little-known API called \code{posix_spawn} that
addresses this problem by creating a process and loading an image
directly (\code{CreateProcess} in Windows is similar).  This is
equivalent to \code{fork}/\code{exec}, but its specification eliminates
the intermediate sub-operations.  As a result, \code{posix_spawn}
commutes with most other operations and permits a broadly scalable
implementation.
% As we show in \cref{sec:eval:app}, using \code{posix_spawn} instead of
% \code{fork} and \code{exec} leads to better scalability.

Another example, \code{stat}, retrieves and returns many
different attributes of a file simultaneously, which makes it
non-commutative with operations on the same file that change any
attribute returned by \code{stat} (such as \code{link}, \code{chmod},
\code{chown}, \code{write}, and even \code{read}).  In practice,
applications invoke \code{stat} for just one or two of the returned
fields. An alternate API that gave applications
control of which field or fields were returned
would commute with more operations and enable a more scalable
implementation of \code{stat}, as we show in
\cref{sec:eval:microbenchmarks}.

POSIX has many other examples of compound return values.
\code{sigpending} returns all pending signals, even if the caller only
cares about a subset;
% \code{wait} returns all children, even if the caller
% just cares about a subset;
and \code{select} returns all ready file
descriptors, even if the caller needs only one ready FD.

\XXX[AC]{Examples of weird return values that distinguish things in
pointless ways?  I think open had something annoying with this.}

\XXX[AC]{Embrace non-determinism?  Embrace under-specification?  Maybe
this is really about naming.  Do not constrain resource names?  Name
resources non-deterministically?  Use opaque resource names?}

\paragraph{Embrace specification non-determinism.}

POSIX's ``lowest available FD'' rule is a classic example of
overly deterministic design that results in poor scalability.
Because of this rule, \code{open} operations in the same process (and
any other FD allocating operations) do not commute, since the order in
which they execute determines the returned FDs.  This constraint is rarely
needed by applications and an alternate interface
that could return any
unused FD would allow FD allocation operations to commute and enable
implementations to use well-known scalable allocation methods.  We will
return to this example, too, in \cref{sec:eval:microbenchmarks}.
Many other POSIX interfaces get this right: \code{mmap} can
return any unused virtual address and \code{creat} can assign any unused
inode number to a new file.

\XXX[AC]{Linearizable interfaces are really the problem below.
Communication interfaces should allow causal or unconstrained ordering.}

\paragraph{Permit weak ordering.}

Another common source of limited commutativity is strict ordering
requirements between operations.  For many operations, ordering is
natural and keeps interfaces simple to use; for example, when one thread
writes data to a file, other threads can immediately read that data.
%
Synchronizing operations like this are naturally non-commutative.
%
Communication interfaces, on the other hand, often enforce strict
ordering, but may not need to.
For instance, most systems order all messages sent via a local Unix
domain socket, even when using
\code{SOCK_DGRAM}, so any \code{send} and \code{recv} system
calls on the same socket do not commute (except in error conditions).
This is often unnecessary, especially in multi-reader or multi-writer
situations, and an alternate interface that does not enforce ordering
would allow \code{send} and \code{recv} to commute as long as there is
both enough free space and enough pending messages on the socket, which
would in turn allow an implementation of Unix domain sockets to support
scalable communication (which we use in \cref{sec:eval:app}).
\XXX[AC]{Should be ``as we demonstrate'', but we don't isolate unordered
sockets.}

\XXX[AC]{Delay?  Decouple and delay?  Enable asynchrony?  Decouple
asynchronous operations?  Maybe this should be part of decompose
compound operations because we're splitting the initiation from the
final action?  Or part of weak ordering because these operations should
be weakly ordered with respect to other cores.  This also involves
specification non-determinism, but in the same way weak ordering does.
Release resources asynchronously?  Really it's about resources that are
coordinated with other cores.  Coordinate asynchronously?  The above is
about coordinate you want to be immediate, just unordered.  This is
about coordination that can be delayed.}

\paragraph{Release resources asynchronously.}

% \XXX[AC]{kill isn't very compelling.  Others are better.}
% Another example of strict ordering is \code{kill}, which requires that
% the signal be immediately delivered to the target (if it is not masked).
% This means \code{kill} does not commute with the target executing
% instructions.  A delayed \code{kill} would commute with target execution
% and other operations, and would
% allow a kernel to batch signal delivery.

A closely related problem is that many POSIX operations have global
effects that must be visible before the operation returns.
This is generally good design for usable interfaces, but for
operations that release resources, this is often stricter than
applications need and expensive to ensure.  For example, writing
to a pipe
must deliver \code{SIGPIPE} immediately if there are no
read FDs for that pipe, so pipe writes do not commute with the last
\code{close}
of a read FD.  This requires aggressively tracking the number of
read FDs; a relaxed specification that promised to eventually deliver the
\code{SIGPIPE} would allow implementations to use more scalable
read FD tracking.
Similarly, \code{munmap} does not commute
with memory reads or writes of the unmapped region from other threads.
Enforcing this requires non-scalable remote TLB shootdowns before \code{munmap}
can return, even though depending
on this behavior usually indicates a bug.  An \code{munmap} (perhaps an
\code{madvise}) that released virtual memory asynchronously
would let the kernel
reclaim physical memory lazily and batch or eliminate remote TLB
shootdowns.

